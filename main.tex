\documentclass{article}
\usepackage[utf8]{inputenc}
\usepackage{natbib}
\usepackage{amssymb}

\title{Transfinite extensions of Hibbard's intelligence measure}
\author{Samuel Alexander\thanks{The U.S.\ Securities and Exchange Commission}}
\date{2020}

\begin{document}

\maketitle

\begin{abstract}
    Fill this in.
\end{abstract}

\section{Introduction}

In his insightful paper \cite{hibbard}, Bill Hibbard introduces a novel
intelligence measure (which we will here refer to as the \emph{classical Hibbard measure})
for agents with artificial general intelligence.
Hibbard's measure is based on ``the game of adversarial sequence prediction
against a hierarchy of increasingly difficult sets of'' evaders (environments that attempt
to emit $1$s and $0$s in such a way as to evade prediction).
The levels of Hibbard's hierarchy are labelled by natural numbers\footnote{Technically,
Hibbard's hierarchy begins at level $1$ and he separately defines what it means for
an agent to have intelligence $0$, but that definition is equivalent to what would result
by declaring that the $0$th level of the hierarchy consists of the empty set of evaders.}, and
an agent's classical Hibbard measure is the maximum $n\in\mathbb N$ such that
said agent can eventually predict all the evaders in the $n$th level of the hierarchy,
or implicitly\footnote{Hibbard does not explicitly include the $\infty$ case in his
definition, but in his Proposition 3 he refers to agents having ``finite intelligence'', and
it is clear from context that by this he means agents who fail to predict some evader
somewhere in the hierarchy.} an agent's classical Hibbard measure is defined to be $\infty$
if said agent can eventually predict all the evaders in all levels of the hierarchy.




\bibliographystyle{plain}
\bibliography{references}
\end{document}
