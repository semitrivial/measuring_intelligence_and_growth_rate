\documentclass{article}
\usepackage[utf8]{inputenc}
\usepackage{natbib}
\usepackage{amssymb}
\usepackage{amsthm}
\usepackage{amsmath}
\usepackage{mathrsfs}
\usepackage{mathdots}
\newtheorem{theorem}{Theorem}
\newtheorem{definition}[theorem]{Definition}
\newtheorem{corollary}[theorem]{Corollary}
\newtheorem{example}[theorem]{Example}
\newtheorem{assumption}[theorem]{Assumption}
\newtheorem{lemma}[theorem]{Lemma}
\newtheorem{problem}[theorem]{Problem}
\newtheorem{proposition}[theorem]{Proposition}
\newtheorem{informallemma}[theorem]{Informal Lemma}
\newtheorem{remark}[theorem]{Remark}
\newtheorem{openquestion}[theorem]{Open Question}

\title{Extending Hibbard's intelligence measure by delegating hard choices to the agent}
\author{Samuel Alexander\thanks{The U.S.\ Securities and Exchange Commission}}
\date{2020}

\begin{document}

\maketitle

\begin{abstract}
When studying the theory of agents with Artificial General Intelligence
(the theory of AGIs), one can sometimes avoid arbitrary decisions by
delegating those decisions to the AGIs under investigation.
In 2011, Hibbard suggested an intelligence measure for AGIs acting as
predictors in adversarial sequence prediction games. Hibbard's measure
depends arbitrarily on a function-sequence
imported from a paper by Liu. We argue that Hibbard's measure is trivial:
any genuine AGI should have Hibbard intelligence $\infty$,
because the functions in Liu's paper are
trivial to a genuine AGI. Our key insight is this: rather than
arbitrarily choosing a function-sequence, we can delegate that choice to an AGI.
In this way, we propose
non-trivial variations of Hibbard's original measure, which we call
relative Hibbard measures.
However, relative Hibbard measures are still flawed. To repair their flaws,
we further propose what we call transfinite Hibbard measures,
which depend on an AGI choosing a slow-growing hierarchy (from mathematical
logic) rather than a function-sequence.
\end{abstract}

\section{Introduction}

\begin{quote}
    Todo:
    \begin{enumerate}
        \item Formal theorem suggesting THM/IOI connection.
        \item More background about fast- and slow-growing hierarchies.
        \item In Example bruteforcenottotallyoptimalexample, we choose a function which
            the AGI does not know. Add a lemma showing that such functions must exist.
        \item A lemma to the effect that if X knows M is a Turing machine which computes
            f at all but finitely many points, then there exists a TM M' which computes
            f everywhere and such that X knows M' is total.
    \end{enumerate}
\end{quote}

At a high-level, this paper is about a meta-technique in theoretical research
on agents with Artificial General Intelligence (that is, on AGIs), namely:
avoiding difficult and arbitrary decisions by delegating those decisions to the
very AGIs we are studying. To speak metaphorically, suppose we were fighting about
what clothes the Oracle of Delphi should wear: such fighting could be avoided
by asking the oracle herself what clothes an Oracle of Delphi should wear. At
a low-level, this paper is about a certain specific intelligence measure, and
improvements thereon: these serve as a kind of proving ground for the
higher-level meta-technique.

In his insightful paper \cite{hibbard}, Bill Hibbard introduces a novel
intelligence measure (which we will here refer to as the \emph{original Hibbard measure})
for AGIs.
Hibbard's measure is based on a game of adversarial sequence
prediction \cite{hibbard2008adversarial}
``against a hierarchy of increasingly difficult sets of'' evaders (environments that attempt
to emit $1$s and $0$s in such a way as to evade prediction).
The levels of Hibbard's hierarchy are labelled by natural numbers\footnote{Technically,
Hibbard's hierarchy begins at level $1$ and he separately defines what it means for
an agent to have intelligence $0$, but that definition is equivalent to what would result
by declaring that the $0$th level of the hierarchy consists of the empty set of evaders.}, and
an agent's original Hibbard measure is the maximum $n\in\mathbb N$ such that
said agent learns to predict all the evaders in the $n$th level of the hierarchy,
or implicitly\footnote{Hibbard does not explicitly include the $\infty$ case in his
definition, but in his Proposition 3 he refers to agents having ``finite intelligence'', and
it is clear from context that by this he means agents who fail to predict some evader
somewhere in the hierarchy.} an agent's original Hibbard measure is $\infty$
if said agent learns to predict all the evaders in all levels of Hibbard's hierarchy.

In this paper, we will argue that the original Hibbard measure is
trivial: we will argue that \emph{all} suitably idealized AGIs should have
original Hibbard intelligence $\infty$. This is because Hibbard's hierarchy
depends on a function-sequence which is (we would argue) arbitrarily chosen,
and which happens to consist of functions which should be trivial to a genuine AGI.

We will suggest two different methods to repair Hibbard's intelligence measure.
\begin{itemize}
    \item
    The first method we describe will be elementary, in that it will
    essentially involve
    no mathematics more advanced than the mathematics used in Hibbard's original approach.
    This will yield a non-trivial family of intelligence measures,
    which we will call \emph{relative Hibbard measures}, but these will still suffer a flaw.
    \item
    The second method we describe will make use of \emph{slow-growing hierarchies}
    \cite{weiermann2002slow} (an advanced notion from mathematical logic) and will
    produce a family of ordinal-number-valued intelligence measures, which we will
    call \emph{transfinite Hibbard measures}, to avoid the above-mentioned flaw.
\end{itemize}
Both of these methods will, rather than producing a single intelligence measure,
rather, produce a family of intelligence measures indexed by a parameter $X$ where
$X$ is an AGI.
Every AGI $X$ will have a corresponding relative Hibbard measure $|\bullet|_X$
(for any AGI $Y$, $|Y|_X$ will be a natural-number-valued measurement of $Y$'s intelligence
``according to $X$'').
Every AGI $X$ will also have a corresponding transfinite Hibbard measure
$\|\bullet\|_X$
(for any AGI $Y$, $\|Y\|_X$ will be a computable-ordinal-valued measurement of $Y$'s
intelligence ``according to $X$'').

The structure of the paper is as follows.
\begin{itemize}
    \item
    In Section \ref{agiperspectivesection}, we present an idealized viewpoint of AGIs.
    \item
    In Section \ref{originalmeasuresection}, we review the original Hibbard measure.
    \item
    In Section \ref{growthratesection}, we argue that Hibbard's idea essentially
    boils down to trying to
    solve a problem that computer scientists have already solved via Big-O notation.
    We propose a Hibbard-style method for quantifying AGI intelligence using Big-O
    notation.
    \item
    In Section \ref{exoticsection}, we consider several other solutions to
    the problem which computer
    scientists solve using Big-O notation, and the Hibbard-style measures which
    correspond to these solutions.
    \item
    In Section \ref{trivialitysection}, we argue that
    under certain idealizing assumptions, \emph{every} AGI probably has
    original Hibbard intelligence $\infty$.
    \item
    In Section \ref{conclusionsection}, we summarize and make concluding remarks.
\end{itemize}

\section{An idealized viewpoint of AGIs}
\label{agiperspectivesection}

\begin{quote}
    ``Thus it is a property of man[-like intelligence] to be capable of learning
    grammar; for if he is a man[-like intelligence], then he is capable of learning
    grammar, and if he is capable of learning grammar, then he is
    a man[-like intelligence].'' ---Aristotle \cite{aristotle} (modified)
\end{quote}

As yet, we are unaware of any formal mathematical definition of what an AGI
actually is. We take the stance that an AGI should be thought of like a faithful
employee which can understand
practical conversation in a common human language (such as
English) and capable of (and faithfully willing to) obey commands
expressed by its employer in said language\footnote{Yampolskiy \cite{yampolskiy2013turing}
nicely captures this idea when he talks about reducing the Turing Test to the
problem of Programming, saying: `Having access to an Oracle capable of solving Programming
allows us to solve TT via a simple reduction. For any statement $S$ presented during
TT transform said statement into a programming assignment of the form: ``Write a
program which would respond to $S$ with a statement indistinguishable from a statement
provided by an average human{''}'.}.
So if the AGI's employer said, ``Recite the digits of pi until I tell you to
stop,'' then the AGI would begin computing and reciting digits of pi, until the
employer told it to stop (if ever). Unlike a human employee, who would eventually
get fed up with such a task, an AGI should have no fear of tedium, and should
never grow tired of performing whatever task it has been assigned (we assume
the AGI never runs out of memory and can indeed run forever, just like a Turing
machine is routinely assumed to be able to run forever and have infinite tape).

We think assumptions similar to the above are implicit in many attempts to study AGIs.
Many authors who set out to study
AGIs actually study not the full AGI, but rather the AGI's persona when it is
performing some general type of task. For example:
\begin{itemize}
    \item
    In the Turing Test, the AGI is reduced to a conversation partner pretending to
    be human.
    An AGI is more than a pretend human conversation partner;
    but an AGI would act as a pretend human conversation partner
    if commanded to do so.
    \item
    In Hibbard's paper \cite{hibbard}, the AGI is reduced to a predictor in an
    adversarial sequence prediction game.
    An AGI is more than a predictor; but an AGI would
    act as such a predictor if commanded to do so.
    \item
    Some authors \cite{legg} \cite{hernandez}
    reduce the AGI to a reinforcement learning agent.
    An AGI is more than a reinforcement
    learning agent; but an AGI would
    act as a reinforcement learning agent if commanded to do so.
    \item
    Some philosophers
    reduce the AGI to a mere \emph{knower} who performs
    no actions but simply knows things in some formal language, or, equivalently,
    to a formal theory (namely the list of all the things it knows).
    An AGI is more than a mere knower
    or theory \cite{wang2007three};
    but an AGI would act as a mere knower (and give rise to a resulting theory)
    if commanded to do so.
    \item
    In the Non-Axiomatic Reasoning System research program (NARS) \cite{nars},
    it is hoped that an AGI can be realized via the implementation of what is,
    essentially, a special type of database capable of being bombarded by
    many (possibly conflicting) assertions in a certain formal language
    and of outputting probabilistic answers to realtime queries about said
    assertions. An AGI is more
    than a mere database; but
    an AGI would act as such a database if commanded to do so.
\end{itemize}
In the same way that IQ tests are intended to measure human intelligence despite
the fact that humans are more than professional IQ-test-takers,
likewise, the above cross-sections of AGI serve to illuminate AGI
even though none of them capture the full AGI. Many of these kind of
cross-sections\footnote{Or rather, the task of doing \emph{good} at these
cross sections---whatever
that means---for anyone can easily program a Turing Test participant that just
repeats the same phrase over and over again, but that is not an example of
doing the Turing Test well. Anyone can program a reinforcement learning agent that
acts pseudo-randomly, but that is not an example of doing reinforcement learning well.}
are probably examples of so-called \emph{AI-Complete} problems
\cite{yampolskiy2012ai} \cite{yampolskiy2013turing}.

\begin{remark}
\label{temporalremark}
    Throughout this paper, whenever we speak of an AGI, we really mean the
    state of a general-artificially intelligent agent's mind at an instant in time.
    To illustrate this, imagine a robot which possesses artificial general intelligence.
    Suppose at time $t$ the robot observes something which updates its knowledge.
    Then we consider the same robot to define at least two distinct AGIs, one AGI
    immediately before time $t$, and another
    immediately after time $t$.
    Whenever we speak of the outputs which an AGI would make in response to some
    command, we implicitly mean the hypothetical outputs which the AGI would make
    if, immediately upon receiving the command, the AGI were isolated from all
    external stimulus.
\end{remark}

Remark \ref{temporalremark} is illustrated by the following
informal lemma.

\begin{informallemma}
\label{teachinglemma}
    If $X$ is an AGI and $\phi$ is a true sentence in the language of Peano arithmetic
    such that $X$ does not know $\phi$, then there exists an AGI $Y$ such that:
    \begin{enumerate}
        \item
        $Y$ knows $\phi$.
        \item
        For every sentence $\psi$ in the language of Peano arithmetic,
        if $X$ knows $\psi$, then $Y$ knows $\psi$.
    \end{enumerate}
    In fact, $Y$ can be obtained from $(X,\phi)$ in an effective way, by which we mean
    that there is a computable function $F$ which takes two inputs $(x,y)$ such that
    if $x$ encodes AGI $X$ and $y$ encodes true sentence $\phi$ of Peano arithmetic
    then $F(x,y)$ encodes AGI $Y$ as above.
\end{informallemma}

\begin{proof}[Informal proof]
    Let $Y$ be the AGI which would result immediately after $X$ were given the command:
    ``Know $\phi$; in other words, henceforward, operate under the assumption
    that $\phi$ is true.'' Then $Y$ knows $\phi$ by construction, and $Y$ knows
    every sentence $\psi$ in the language of Peano arithmetic that $X$ knows
    (because the indicated command should not cause $X$ to suddenly doubt any
    previously known arithmetical facts---note that since $\phi$ is true, $X$
    did not previously know $\neg\phi$, or anything else inconsistent with $\phi$,
    since knowledge is truthful). This process is clearly effective, so the second
    part of the Informal Lemma follows by Church's Thesis.
\end{proof}

Note that part (2) of Informal Lemma \ref{teachinglemma} might fail if
$\psi$ were not restricted to purely arithmetical formulas.
For example, assume $P\neq NP$ and let $\phi$ express $P\neq NP$.
It could very well be that $X$ knows that ``I do not know $\phi$''---i.e., that
$X$ knows ``$\neg K(\phi)$'' (a formula in the language of Epistemic Arithmetic
\cite{shapiro})---but that $Y$, having been taught that $\phi$ is true,
no longer knows ``I do not know $\phi$''.

\begin{assumption}
\label{idealizingassumption}
    We make the following idealizing assumptions about every AGI $X$.
    \begin{itemize}
        \item
        (Basic mathematical exposure) $X$ knows the axioms of ZFC (the axiomatic
        system underlying mainstream mathematics).
        \item
        (Mathematical truthfulness) If $X$ knows some statement in the language
        of ZFC, then that statement is true.
        \item
        (Mathematical reasoning ability) If $X$ knows some set $S$ of mathematical
        axioms then, given sufficient time to think about it, $X$ would eventually
        become aware of all the consequences of $S$.
        \item
        (Caution) $X$ avoids getting stuck: If $X$ is commanded to carry out any
        task which involves outputting things
        forever, and if the task is not impossible
        (e.g., does not require solving the Halting Problem or doing other
        non-computable or nonsensical things),
        then $X$ will continue outputting things (perhaps
        after long intervals of time between outputs) indefinitely.
        \item
        (Mechanicalness) If $X$ is commanded to carry out a task
        which involves outputting things forever, and if those outputs depend only on
        $X$'s own decisions (not on any external stimulus),
        then the resulting outputs will
        be computably enumerable (in other words, some Turing machine could
        generate said outputs).
    \end{itemize}
\end{assumption}

Concerning the \emph{mechanicalness} part of Assumption \ref{idealizingassumption},
it is important to keep in mind that although the assumption guarantees that if $C$
is any such command, there exists a Turing machine $M$ which computes enumerates
$X$'s outputs in response to command $C$, nevertheless:
\begin{enumerate}
    \item
    It is not necessarily the case that $X$ knows that $M$
    enumerates an infinite list.
    \item
    Even if $X$ does know that $M$ enumerates an infinite list, it is not necessarily
    the case that $X$ knows that the list enumerated by $M$ is the list of things
    $X$ would output in response to command $C$.
\end{enumerate}
We will give two examples to illustrate the above two caveats.

\begin{definition}
\label{functionsuccdefn}
For any $f,g:\mathbb N\to\mathbb N$, write $f\succ g$ as shorthand for the statement
that there is some $i\in\mathbb N$ such that for all $x\geq i$, $f(x)>g(x)$.
\end{definition}

\begin{example}
\label{counterintuitiveexample1}
    Suppose $X$ is commanded: ``Output Turing machines $M_0,M_1,\ldots$
    such that each $M_i$ computes a total function $f_i:\mathbb N\to\mathbb N$,
    and such that each $f_{i+1}\succ f_i$, and such that for every Turing machine $M$
    such that you know $M$ computes a total function $f:\mathbb N\to\mathbb N$,
    there exists some $i$ such that $f_i>f$.'' By \emph{mechanicalness}, there
    exists a Turing machine $T$ which enumerates $M_0,M_1,\ldots$.
    But $X$ does not know that $T$ enumerates an infinite list.
\end{example}

\begin{proof}
    Assume (for sake of contradiction) that $X$ knows that $T$ enumerates an
    infinite list. Since $X$ is capable of basic mathematical reasoning, it follows
    that $X$ knows that Turing machine $M$ defines a total function $f:\mathbb N\to\mathbb N$,
    there $M$ is the Turing machine which takes input $n\in\mathbb N$ and proceeds as
    follows:
    \begin{enumerate}
        \item
        Run $T$ until it has outputted Turing machines $M_0,\ldots,M_n$. (If $T$
        ever outputs something that is not a Turing machine, halt and output $0$.)
        \item
        For each $i\leq n$, run $M_i$ on input $n$ until it outputs a natural
        number $m_{n,i}$.
        \item
        Output $\max\{m_{n,0},\ldots,m_{n,n}\}+1$.
    \end{enumerate}
    Above, since each $M_i$ computes $f_i$, it follows that each $m_{n,i}=f_i(n)$.
    Thus, for each $i$, $f\succ f_i$, because for all $x\geq i$,
    $f(x)=\max\{m_{x,0},\ldots,m_{x,x}\}+1\geq m_{x,i}+1=f_i(x)+1$.
    But by assumption, since $X$ knows that $M$ computes a total function from
    $\mathbb N$ to $\mathbb N$, there is some $i$ such that
    $f_i\succ f$. So $f\succ f_i$ and $f_i\succ f$, a contradiction.
\end{proof}

\begin{example}
\label{counterintuitiveexample2}
    (G\"odel's incompleteness theorem)
    Assume $X$ knows that its own knowledge is truthful.
    Suppose $X$ is commanded by command $C$:
    ``Output all the true statements (in the language of
    Peano arithmetic) that you know are true.'' Let the resulting statements be
    $\phi_0,\phi_1,\ldots$. By \emph{mechanicalness}, there is a Turing machine $M$
    such that $M$ enumerates $\phi_0,\phi_1,\ldots$. Then $X$ does not know
    ``$M$ enumerates the list of things I would output in response to $C$''.
\end{example}

\begin{proof}
    Assume (for sake of contradiction) that $X$ knows ``$M$ enumerates the
    list of things I would output in response to $C$''.
    Then, since $X$ knows that its own knowledge is truthful,
    $X$ knows that $M$ enumerates a list of true statements of Peano arithmetic
    (because $X$ knows that it knows all statements output in response to $C$).
    Of course, this fact $\Phi$---that $M$ enumerates a list of true statements of Peano
    arithmetic---is not itself expressible in Peano arithmetic, by Tarski's
    undefinability theorem \cite{tarski1936wahrheitsbegriff},
    and therefore this fact need not be one of the $\phi_i$'s.
    However, $\Phi$ implies (and hence $X$ should know)
    a weaker fact $\Phi'$, namely, that $M$ enumerates a consistent
    list of statements of Peano Arithmetic. This consistency statement $\Phi'$ \emph{is}
    expressible in Peano arithmetic, and thus since $X$ knows it,
    there is some $i$ such that $\Phi'=\phi_i$. Thus, $T=\{\phi_0,\phi_1,\ldots\}$
    is a computably enumerable
    theory (in the language of Peano arithmetic) which is true, and which
    includes the axioms of Peano arithmetic (since $X$ knows those), and which
    includes a statement $\Phi'$ stating its own consistency. This is a contradiction
    because it violates G\"odel's second incompleteness theorem.
\end{proof}

\section{Hibbard's original measure}
\label{originalmeasuresection}

Hibbard's intelligence measure is based on the AGI's performance against
many opponents in a game of adversarial sequence prediction (we define this
formally below). The AGI acts as
a predictor $p$ which competes against evaders $e$. In each step of the game,
both predictor and evader simultaneoulsy choose a binary digit, $1$ or $0$.
Only after both of them have made their choice do they see which choice the other
one made, and then the game proceeds to the next step. The predictor's goal in
each step of the game is to choose the same digit that the evader will choose;
the evader's goal is to choose a different digit than the predictor. The predictor
wins the game (and is said to \emph{learn to predict $e$}) if, after finitely many
initial steps, eventually the predictor always chooses the same digit as the
evader.

\begin{definition}
By $B$, we mean the binary alphabet $\{0,1\}$. By $B^*$, we mean the set of all
finite binary sequences. By $\langle\rangle$ we mean the empty binary sequence.
\end{definition}

\begin{definition}
\label{evaderpredictordefn}
    (Predictors and evaders)
    \begin{enumerate}
        \item
        By a \emph{predictor}, we mean a Turing machine $p$
        which takes as input a finite (possibly empty) binary sequence
        $(x_1,\ldots,x_n)\in B^*$
        (thought of as a sequence of \emph{evasions})
        and outputs $0$ or $1$ (thought of as a \emph{prediction}), which output
        we write as $p(x_1,\ldots,x_n)$.
        \item
        By an \emph{evader}, we mean a Turing machine $e$
        which takes as input a finite (possibly empty) binary sequence
        $(y_1,\ldots,y_n)\in B^*$
        (thought of as a sequence of \emph{predictions})
        and outputs $0$ or $1$ (thought of as an \emph{evasion}), which output
        we write as $e(y_1,\ldots,y_n)$.
        \item
        For any predictor $p$ and evader $e$, the \emph{result of $p$ playing the
        game of adversarial prediction against $e$} (or more simply, the \emph{result of
        $p$ playing against $e$}) is the infinite binary sequence
        $(x_1,y_1,x_2,y_2,\ldots)$
        defined as follows:
        \begin{enumerate}
            \item
            The first evasion
            $x_1=e(\langle\rangle)$ is
            the output of $e$ when run on the empty prediction-sequence.
            \item
            The first prediction
            $y_1=p(\langle\rangle)$ is
            the result of applying $p$ to the empty evasion-sequence.
            \item
            For all $n>0$, the $(n+1)$th evasion
            $x_{n+1}=e(y_1,\ldots,y_n)$ is
            the output of $e$ on the sequence of the first $n$ predictions.
            \item
            For all $n>0$, the $(n+1)$th prediction
            $y_{n+1}=p(x_1,\ldots,x_n)$ is
            the result of applying $p$ to the first $n$ evasions.
        \end{enumerate}
        \item
        Suppose $r=(x_1,y_1,x_2,y_2,\ldots)$ is the result of a predictor $p$ playing
        against an evader $e$. For every $n\geq 1$,
        we say \emph{the predictor wins round $n$ in $r$}
        if $x_n=y_n$; otherwise,
        \emph{the evader wins round $n$ in $r$}.
        We say that \emph{$p$ learns to predict $e$}
        (or simply that \emph{$p$ learns $e$}) if there is some $N\in\mathbb N$
        such that for all $n>N$, $p$ is the winner of round $n$ in $r$.
    \end{enumerate}
\end{definition}

Note that if $e$ simply ignores its inputs $(y_1,\ldots,y_n)$ and instead
computes $e(y_1,\ldots,y_n)$ based only on $n$, then $e$ is essentially a sequence.
Thus Definition \ref{evaderpredictordefn} is a generalization of sequence prediction,
which many authors have written about (such as \cite{legg2006there}, who gives many
references).

The following definition makes explicit a subtlety which is implicit in Hibbard's
original paper.

\begin{definition}
\label{Xspredictordefn}
    Suppose $X$ is an AGI. By \emph{$X$'s predictor}, we mean the predictor which
    $X$ would act as if $X$ were commanded to act as a predictor as in
    Definition \ref{evaderpredictordefn}.
\end{definition}

In the following definition, we differ from Hibbard's original paper
because of a minor (and fortunately, easy-to-fix) error there.

\begin{definition}
\label{tsubedefinition}
    Suppose $M$ is any Turing machine which computes a total function from $B^*$ to $B$.
    For any $n\in\mathbb N$, let $T_M(e)$ be the maximum number of steps that $M$ takes
    to run on any length-$n$ sequence of binary digits.
    In other words, $t_M(0)$ is the number of steps $M$ takes to run on $\langle\rangle$,
    and for all $n>0$,
    \[
        t_M(n) = \max_{b_1,\ldots,b_n\in \{0,1\}}
        (\text{number of steps $M$ takes to run on $(b_1,\ldots,b_n)$}).
    \]
\end{definition}

\begin{example}
    Let $e$ be an evader. Then
    $t_e(2)$ is equal to the number of steps $e$ takes to run on input
    $(0,0)$, or to run on input $(0,1)$, or to run on input $(1,0)$, or to run on input
    $(1,1)$---whichever of these four possibilities is largest.
\end{example}

\begin{definition}
\label{evadersetdefinition}
    (The evader-set bounded by a function)
    Suppose $f:\mathbb N\to\mathbb N$. We define the \emph{evader-set bounded by $f$},
    written $E_f$, to be the set of all evaders $e$ such that
    there is some $k\in\mathbb N$ such that $\forall n>k$,
    $t_e(n)<f(n)$.
\end{definition}

In the following broad definition, for every function-sequence of functions
from $\mathbb N$ to $\mathbb N$, we define a corresponding intelligence measure
for AGIs.
The original Hibbard measure is a special case for one specific
such function-sequence.

\begin{definition}
\label{generalintelligencemeasuredefn}
    (The general Hibbard measure given by a function-sequence)
    Suppose $(g_1,g_2,\ldots)$ is a function-sequence,
    each $g_i:\mathbb N\to\mathbb N$.
    For each $m\in\mathbb N$, define $f_m:\mathbb N\to\mathbb N$ by
    \[f_m(k)=\max_{0<i\leq m}\max_{j\leq k}g_i(j).\]
    For any AGI $X$, we define the \emph{Hibbard intelligence of $X$
    given by $(g_1,g_2,\ldots)$}, written $|X|_{(g_1,g_2,\ldots)}$, as follows.
    Let $p$ be $X$'s predictor (Definition \ref{Xspredictordefn}).
    We define $|X|_{(g_1,g_2,\ldots)}$ to be the maximum $m>0$
    such that $p$ learns to predict $e$ for every $e\in E_{f_m}$
    (or $0$ if there is no such $m$, or $\infty$ if $p$ learns to predict $e$
    for every $e\in E_{f_m}$ for every $m$).
\end{definition}

\begin{definition}
\label{classichibbardmeasuredefn}
    The \emph{original Hibbard measure} is defined to be the Hibbard intelligence
    measure given by one specific family $(g_1,g_2,\ldots)$ of functions, namely:
    Liu's \cite{liu1960enumeration} enumeration of the primitive recursive
    functions.
\end{definition}


\section{Quantifying growth rates of functions}
\label{growthratesection}

The following is a very general and open-ended problem.

\begin{problem}
\label{bigoproblem}
    Quantify the growth-rate of functions from $\mathbb N$ to $\mathbb N$.
\end{problem}

The standard solution to Problem \ref{bigoproblem} in computer science is to quantify
growth rates of arbitrary functions by comparing them to more familiar functions using
Big-O notation, Big-$\Omega$ notation, or Big-$\Theta$ notation.
Knuth defines \cite{knuth1976big} these as follows (we modify the definition slightly because
we are only concerned here with functions from $\mathbb N$ to $\mathbb N$).

\begin{definition}
\label{bigodefn}
    Suppose $f:\mathbb N\to\mathbb N$. We define the following the following function-sets.
    \begin{itemize}
        \item
        $O(f(n))$ is the set of all $g:\mathbb N\to\mathbb N$ such that
        there is some real $C>0$ and some natural number $n_0$ such that
        for all $n\geq n_0$, $g(n)\leq Cf(n)$.
        \item
        $\Omega(f(n))$ is the set of all $g:\mathbb N\to\mathbb N$ such that
        there is some real $C>0$ and some natural $n_0$ such that
        for all $n\geq n_0$, $g(n)\geq Cf(n)$.
        \item
        $\Theta(f(n))$ is the set of all $g:\mathbb N\to\mathbb N$ such that
        there are some real $C>0$ and $C'>0$ and some natural $n_0$ such that
        for all $n\geq n_0$, $Cf(n)\leq g(n)\leq C'f(n)$.
    \end{itemize}
\end{definition}

The original Hibbard measure (Definition \ref{classichibbardmeasuredefn}) measures
intelligence by seeing how far along a certain specific function-sequence we must go
before we reach functions which grow so fast that an AGI's predictor does not learn
all evaders whose computation-time grows so fast. This strategy could be applied to
Problem \ref{bigoproblem} as well, and would yield a solution, but a solution which we
would argue is not very useful. Namely, adapting Hibbard's measure to
Problem \ref{bigoproblem} would yield the following solution.

\begin{definition}
\label{originalhibbardcomplexitydefn}
    Suppose $f:\mathbb N\to\mathbb N$.
    Let $g_1,g_2,\ldots$ be as in Definition \ref{classichibbardmeasuredefn}, and let
    $f_1,f_2,\ldots$ be as in Definition \ref{generalintelligencemeasuredefn}.
    We define the \emph{original Hibbard complexity}
    of $f$ to be the maximum $n$ such that $f_n\not\succ f$. If there is no such
    $n$, then we say $f$ \emph{has original Hibbard complexity $\infty$}.
\end{definition}

Definition \ref{originalhibbardcomplexitydefn} would clearly not be much use in
practice. It is much more useful to say that an algorithm has, say, asymptotic
runtime $O(n^3)$ than to say that said algorithm's runtime has original Hibbard
complexity (say) $835$. If someone were told that an algorithm's runtime had
original Hibbard complexity (say) $835$, that number would be totally meaningless
to them except as a key which they could use to look up which function $f_{835}$
happens to be (in Definition \ref{originalhibbardcomplexitydefn}). The number
$835$ would merely serve to obfuscate, it would merely play the role of a worthless
middleman.

Having made the above observations, it becomes quite clear how Hibbard's original
idea can be imporved upon to become more useful, more practical, and more familiar
to computer scientists.

\begin{definition}
\label{bigointelligencedefn}
    Suppose $X$ is an AGI, and suppose $f:\mathbb N\to\mathbb N$.
    \begin{itemize}
        \item
        We say \emph{$X$ has Hibbard intelligence $O(f(n))$} if for
        every $g\in O(f(n))$, $X$'s predictor learns every evader in
        $E_g$.
        \item
        We say \emph{$X$ has Hibbard intelligence $\Omega(f(n))$} if for
        every $g\in \Omega(f(n))$, $X$'s predictor learns every evader in
        $E_g$.
        \item
        We say \emph{$X$ has Hibbard intelligence $\Theta(f(n))$} if for
        every $g\in\Theta(f(n))$, $X$'s predictor learns every evader in $E_g$.
    \end{itemize}
\end{definition}

At first glance, Definition \ref{bigointelligencedefn} might seem inferior to
Definition \ref{classichibbardmeasuredefn} because
Definition \ref{classichibbardmeasuredefn} assigns \emph{numerical} intelligence levels.
However, as we pointed out above, those numbers are meaningless except as indices for
a dictionary-lookup making them equivalent to (a more limited version of)
Definition \ref{bigointelligencedefn}. We could imagine someone declaring:
\begin{quote}
    From now on, $f(n)=n$ is complexity $5$, $f(n)=n^2$ is complexity $6$,
    $f(n)=2^n$ is complexity $203$,
    $f(n)=n!$ is complexity $8022$, ..., and from now on, instead of saying a
    function is $\Theta(n)$, we will say that function has growth rate $5$,
    and instead of saying a function is $\Theta(n^2)$, we will say that function
    has growth rate $6$, and instead of saying a function is $\Theta(2^n)$, we will
    say that function has growth rate $203$, and instead of saying a function is
    $\Theta(n!)$, we shall say that function has growth rate $8022$, and ...
\end{quote}
Superficially, this would ``improve'' upon Big-$\Theta$ by giving a more quantitative
solution to Problem \ref{bigoproblem} in one sense, however, it is clear
that this would just be sleight of hand and there would be no actual improvement.

The reader might object that the multiplicative constants $C$ and $C'$ in
Definition \ref{bigodefn} (and implicitly in Definition \ref{bigointelligencedefn})
have no counterpart in the definition of the original Hibbard measure
(Definition \ref{classichibbardmeasuredefn}). However, since AGIs are assumed to
be capable of mathematical reasoning (Assumption \ref{idealizingassumption}), it
follows that if an AGI is smart enough to know that it is safe to run Turing
machines for $f(n)$ steps, then that AGI should also be smart enough to know that
it is safe to run Turing machines for $Cf(n)$ steps.

\section{Hibbard measures using extended number systems}
\label{exoticsection}

We have argued above that it is silly to try to replace asymptotic complexity classes
with natural numbers. In \cite{alexander2020archimedean}, we go further, and cite
asymptotic runtime complexity as an example of a \emph{non-Archimedean} measure which,
we argue, implies that it is silly to replace asymptotic complexity classes
even with \emph{real} numbers. There are, however, non-Archimedean number systems
flexible enough to measure some or all growth rates, and these alternative solutions to
Problem \ref{bigoproblem} would in turn yield Hibbard-style intelligence measures taking
values from said non-Archimedean number systems.
\begin{itemize}
    \item
    Hyperreal numbers, studied in the field of nonstandard analysis,
    are equivalence classes of infinite sequences of reals,
    so for every sequence $(r_0,r_1,r_2,\ldots)$ of reals, there is a corresponding
    hyperreal represented by that sequence.
    Thus, there is a natural and elegant way to solve Problem \ref{bigoproblem}
    using hyperreal numbers. Namely: the growth rate of $f(n)$ is the hyperreal number
    represented by $(f(0),f(1),f(2),\ldots)$. This solution to Problem \ref{bigoproblem}
    will yield an approximate Hibbard-style intelligence measure similar to
    Definition \ref{bigointelligencedefn} (the reason this only yields an approximate
    Hibbard-style intelligence measure is because the hyperreals are not complete).
    \item
    The hyperreal numbers can be embedded into the surreal numbers.
    Thus, for any such embedding,
    there is another natural and elegant way to solve Problem \ref{bigoproblem},
    namely: the growth rate of $f(n)$ is the surreal number corresponding to the
    hyperreal number represented by $(f(0),f(1),f(2),\ldots)$ under the embedding.
    The advantage of the surreal numbers is that they are complete, which will allow
    the corresponding Hibbard-style measure to be exact, not approximate.
    \item
    Another natural way to measure growth rate is using majorization hierarchies
    (such as the slow-growing hierarchy or the fast-growing hierarchy) from mathematical
    logic. A majorization hierarchy assigned ordinal number values to growth rates of
    functions (but not to all functions---for any particular majorization hierarchy, there
    are functions which grow too fast for that hierarchy). These yield corresponding
    Hibbard-style intelligence measures which are ordinal-number-valued.
    \item
    In order to reduce the arbitrary limitations of traditional majorization
    hierarchies, we could delegate the problem of defining a majorization hierarchy to
    an AGI $X$. This leads to a family of Hibbard-style measures---one measure
    $|\bullet|_X$ for each AGI $X$, so that for any particular AGI $X$, for any
    AGI $Y$, there is a corresponding intelligence measure $|Y|_X$ which might be
    thought of as ``$Y$'s intelligence as judged by $X$''.
\end{itemize}

\subsection{Hyperreal Hibbard intelligence}

The so-called \emph{ultrapower construction} of the hyperreals depends on an object
called a \emph{free ultrafilter}, which is a set of subsets of $\mathbb N$ satisfying
certain requirements. It is not important for the purposes of this paper to define what
a free ultrafilter is here. The only important things to know are:
\begin{itemize}
    \item
    A free ultrafilter $\mathscr U$ provides a notion of what it means for a subset
    $S\subseteq\mathbb N$ to be ``large''. Namely: $S$ is considered to be ``large''
    if and only if $S\in \mathscr U$.
    \item
    Mathematical logicians have proven that free ultrafilters exist, but that it is
    impossible to constructively exhibit one. This makes definitions based on free
    ultrafilters non-constructive, and computationally impractical.
\end{itemize}

Because of the computational impracticality of free ultrafilters, the following notion
is also computationally impractical. However, it could potentially be useful for
proving theoretical properties about AGIs.

\begin{definition}
\label{hyperrealhibbardintelligencedefn}
    (Hyperreal Hibbard intelligence)
    Suppose $r$ is a hyperreal number. We say that an AGI $X$ has \emph{hyperreal
    Hibbard intelligence $\geq r$} if the following property holds.
    For every $f:\mathbb N\to\mathbb N$, if the hyperreal represented by
    $(f(0),f(1),f(2),\ldots)$ is $\leq r$, then $X$'s predictor learns every
    evader in $E_f$.
\end{definition}

\subsection{Surreal Hibbard intelligence}

The surreal number system is a number system discovered by John Conway. It is even
more flexible than the hyperreal number system.
There are many ways to embed the hyperreals into the surreals, and there is no
standard or canonical way which stands out. In this subsection, fix some embedding
of the hyperreals into the surreals.

A key property of the surreals
is that they are complete. Thus, given any set $S$ of surreals with a surreal upper bound,
there is a \emph{least} surreal upper bound of $S$, called the \emph{supremum} of $S$
(written $\sup S$). This allows for a Hibbard-style intelligence measure which is more
exact than Definition \ref{hyperrealhibbardintelligencedefn}.

\begin{definition}
    (Surreal Hibbard intelligence)
    Suppose $X$ is an AGI. Let $S_0$ be the set of all functions $f:\mathbb N\to\mathbb N$
    such that $X$'s predictor learns every evader in $E_f$.
    Let $S$ be the set of hyperreals corresponding to members of $S_0$ (that is,
    for each $f\in S_0$, $S$ contains the hyperreal represented by $(f(0),f(1),f(2),\ldots)$).
    We define the \emph{surreal Hibbard intelligence} of $X$ to be
    $\sup S$.
\end{definition}

\subsection{Hibbard intelligence and the traditional majorization hierarchies}

Majorization hierarchies \cite{weiermann2002slow}
provide ordinal-number-valued measures for the growth
rate of certain functions. A majorization hierarchy depends
on many infinite-dimensional parameters. For illustrative purposes,
we will describe the slow-growing hierarchy up to the ordinal $\epsilon_0$,
using standard choices for the parameters.

\begin{definition}
    (Classification of ordinal numbers)
    Ordinal numbers are divided into three types:
    \begin{enumerate}
        \item Zero: The ordinal $0$.
        \item Successor ordinals: Ordinals of the form $\alpha+1$ for some ordinal $\alpha$.
        \item Limit ordinals: Ordinals which are not successor ordinals nor $0$.
    \end{enumerate}
\end{definition}

For example, the smallest infinite ordinal, $\omega$, is a limit ordinal. It is not zero,
nor can it be a successor ordinal, because if it were a successor ordinal, say, $\alpha+1$,
then $\alpha$ would be finite (since $\omega$ is the \emph{smallest} infinite ordinal),
but then $\alpha+1$ would be finite as well.

The ordinal $\epsilon_0$ is the smallest ordinal bigger than the ordinals
$\omega,\omega^\omega,\omega^{\omega^\omega},\ldots$. It satisfies the equation
$\epsilon_0=\omega^{\epsilon_0}$ and can be intuitively thought of as
\[
    \epsilon_0 = \omega^{\omega^{\omega^{iddots}}}.
\]
Ordinals below $\epsilon_0$ include such ordinals as $\omega$,
$\omega^{\omega+1}+\omega^{\omega}+\omega^5+3$,
\[
\omega^{\omega^{\omega^{\omega^\omega}}}+
\omega^{\omega^{\omega^\omega}+\omega^{\omega\cdot 2+1}+\omega^4 + 3}
+ \omega^{\omega^5+\omega^3}+\omega^8+1,
\]
and so on.
Any ordinal below $\epsilon_0$ can be uniquely written in the form
\[
    \omega^{\alpha_1}+\omega^{\alpha_2}+\cdots + \omega^{\alpha_k}
\]
where $\alpha_1\geq\cdots\geq\alpha_k$ are smaller ordinals below $\epsilon_0$---this form
for an ordinal below $\epsilon_0$ is called its \emph{Cantor normal form}.
For example, the Cantor normal form for $\omega^{\omega\cdot 2}\cdot 2+\omega\cdot 3+2$
is
\[
\omega^{\omega\cdot 2}\cdot 2+\omega\cdot 3+2
=
\omega^{\omega\cdot 2} + \omega^{\omega\cdot 2} + \omega^1 + \omega^1 + \omega^1
+\omega^0 + \omega^0.
\]

\begin{definition}
    (Standard fundamental sequences for limit ordinals $\leq\epsilon_0$)
    Suppose $\lambda$ is a limit ordinal $\leq\epsilon_0$. We define a
    \emph{fundamental sequence for $\lambda$},
    written $(\lambda[0],\lambda[1],\lambda[2],\ldots)$, inductively as follows.
    \begin{itemize}
        \item
        If $\lambda=\epsilon_0$, then $\lambda[0]=0$,
        $\lambda[1]=\omega^0$, $\lambda[2]=\omega^{\omega^0}$, and so on.
        \item
        If $\lambda$ has Cantor normal form
        $\omega^{\alpha_1}+\cdots+\omega^{\alpha_k}$ where $k>1$,
        then
        each
        \[
            \lambda[i] = (\omega^{\alpha_1}+\cdots+\omega^{\alpha_{k-1}})
            + (\omega^{\alpha_k})[i].
        \]
        \item
        If $\lambda$ has Cantor normal form $\omega^0$, then each $\lambda[i]=i$.
        \item
        If $\lambda$ has Cantor normal form $\omega^{\alpha+1}$,
        then each $\lambda[i]=\omega^{\alpha}\cdot i$.
        \item
        If $\lambda$ has Cantor normal form $\omega^{\lambda_0}$ where $\lambda_0$
        is a limit ordinal, then each $\lambda[i]=\omega^{\lambda_0[i]}$.
    \end{itemize}
\end{definition}

\begin{example}
    (Fundamental sequence examples)
    \begin{itemize}
        \item
        The fundamental sequence for $\lambda=\omega^5$ is
        $0,\omega^4,\omega^4\cdot 2,\omega^4\cdot 3,\ldots$.
        \item
        The fundamental sequence for $\lambda=\omega^\omega$ is
        $\omega^0,\omega^1,\omega^2,\ldots$.
        \item
        The fundamental sequence for $\lambda=\omega^\omega+\omega$ is
        $\omega^\omega+0,\omega^\omega+1,\omega^\omega+2,\ldots$.
    \end{itemize}
\end{example}

\begin{definition}
\label{slowgrowinghierarchydefn}
    (The standard slow-growing hierarchy up to $\epsilon_0$)
    We define functions $g_\beta:\mathbb N\to\mathbb N$ (for all ordinals
    $\beta\leq \epsilon_0$) by transfinite induction as follows.
    \begin{itemize}
        \item
        $g_0(n)=0$.
        \item
        $g_{\alpha+1}(n) = g_\alpha(n) + 1$.
        \item
        $g_{\lambda}(n) = g_{\lambda[n]}(n)$ if $\lambda$ is a limit ordinal.
    \end{itemize}
\end{definition}

Here are some early levels in the slow-growing hierarchy, spelled out in detail.

\begin{example}
\label{highdetailslowgrowingexample}
    (Early examples of functions in the slow-growing hierarchy)
    \begin{enumerate}
        \item
        $g_1(n)=g_{0+1}(n)=g_0(n)+1=0+1=1$.
        \item
        $g_2(n)=g_{1+1}(n)=g_1(n)+1=1+1=2$.
        \item
        More generally, for all $m\in\mathbb N$,
        $g_m(n)=m$.
        \item
        $g_\omega(n)=g_{\omega[n]}(n)=g_n(n)=n$.
        \item
        $g_{\omega+1}(n)=g_{\omega}(n)+1=n+1$.
        \item
        $g_{\omega+2}(n)=g_{\omega+1}(n)+1=(n+1)+1=n+2$.
        \item
        More generally, for all $m\in\mathbb N$,
        $g_{\omega+m}(n)=n+m$.
        \item
        $g_{\omega\cdot 2}(n)=g_{(\omega\cdot 2)[n]}(n)
        =g_{\omega+n}(n)=n+n=n\cdot 2$.
    \end{enumerate}
\end{example}

Following Example \ref{highdetailslowgrowingexample}, the reader should be able
to fill in the details in the following example.

\begin{example}
    (More examples from the slow-growing hierarchy)
    \begin{enumerate}
        \item
        $g_{\omega^2}(n)=n^2$.
        \item
        $g_{\omega^3}(n)=n^3$.
        \item
        $g_{\omega^\omega}(n)=n^n$.
        \item
        $g_{\omega^{\omega\cdot 3+1}+\omega+5}(n)=n^{3n+1}+n+5$.
        \item
        $g_{\omega^{\omega^{\omega}}}=n^{n^n}$.
    \end{enumerate}
\end{example}

What about $g_{\epsilon_0}$? Thinking of $\epsilon_0$ as
$\omega^{\omega^{\omega^{\iddots}}}$,
one might expect $g_{\epsilon_0}(n)$ to be $n^{n^{n^{\iddots}}}$, but such an infinite tower
of exponents makes no sense if $n>1$. Instead, the answer defies familiar mathematical
notation.

\begin{example}
\label{epsilon0example}
(Level $\epsilon_0$ in the slow-growing hierarchy)
The values of $g_{\epsilon_0}$ are as follows:
\begin{itemize}
    \item
    $g_{\epsilon_0}(0)=0$.
    \item
    $g_{\epsilon_0}(1)=1^1$.
    \item
    $g_{\epsilon_0}(2)=2^{2^2}$.
    \item
    $g_{\epsilon_0}(3)=3^{3^{3^3}}$.
    \item
    And so on.
\end{itemize}
\end{example}

Example \ref{epsilon0example} illustrates how the slow-growing hierarchy can systematically
lead to fast-growing computable functions. By extending the slow-growing hierarchy to
larger ordinals (and choosing appropriate fundamental sequences for those larger ordinals),
one can obtain stupendously fast-growing functions in this way. These in turn serve as
a partial solution to Problem \ref{bigoproblem}: we can say that the growth rate of
an arbitrary function $f:\mathbb N\to\mathbb N$
is equal to $\alpha$, where $\alpha$ is the smallest ordinal such
that $g_\alpha\succ f$ (or $\infty$ if no such $\alpha$ exists).
This in turn provides a corresponding Hibbard-style measure.

\begin{definition}
    If $X$ is an AGI, the \emph{Hibbard measure of $X$ given by the standard
    slow-growing hierarchy up to $\epsilon_0$} is defined to be the
    the supremum of all ordinals $\alpha\leq\epsilon_0$ such that
    $X$'s predictor learns all evaders in $E_{g_\alpha}$ (or $\infty$ if
    $X$'s predictor learns all evaders in $E_{g_\alpha}$ for all $\alpha\leq\epsilon_0$).
\end{definition}

\subsection{Generalized majorization hierarchies}

There are many other majorization hierarchies besides the slow-growing hierarchy.
In Definition \ref{slowgrowinghierarchydefn},
there is nothing special about defining $g_{\alpha+1}(n)=g_{\alpha}(n)+1$.
Many other definitions for $g_{\alpha+1}$ would work, provided $g_{\alpha+1}$ ends
up being faster-growing than $g_\alpha$.
For example, in the so-called \emph{fast-growing hierarchy},
$g_{\alpha+1}(n)$ is defined to be $g^n_\alpha(n)$, where the $g^n_\alpha$
denotes the result of iterating $g_\alpha$ $n$ times, that is,
$g^1_\alpha(n)=g_\alpha(n)$, $g^2_\alpha(n)=g_\alpha(g_\alpha(n))$,
$g^3_\alpha(n)=g_\alpha(g_\alpha(g_\alpha(n)))$, and so on.
In the early levels, this produces much faster-growing functions much more quickly,
but astonishingly, at sufficiently high ordinals, the slow-growing hierarchy
actually catches up with the fast-growing hierarchy \cite{girard1981pi12}
(a beautiful illustration of how counter-intuitive large ordinal numbers can be).
One of the earliest majorization hierarchies is the Hardy
hierarchy \cite{hardy1904theorem}, where $g_{\alpha+1}(n)=g_\alpha(n+1)$.

Likewise, there is also much flexibility in choosing fundamental sequences.
For small ordinals such as the ordinals below $\epsilon_0$, there are very clear
canonical fundamental sequences, but the larger the ordinals become, the harder
it becomes to single out any choice of fundamental sequences as canonical.
And even at low levels, choosing non-canonical fundamental sequences can
drastically alter the resulting majorization hierarchy \cite{weiermann1997sometimes}.

In short, there is no consensus at all about which majorization hierarchies are
most canonical. If anything, there is consensus that there is no consensus.
Fortunately, we are working here in a context where we can refer to AGIs.
Rather than attempt the futile task of choosing a canonical majorization hierarchy
ourselves, we can instead delegate the task to an AGI.


\begin{definition}
\label{fairsequencedefn}
    Suppose $X$ is an AGI. By $h^X_1,h^X_2,\ldots$, we mean the total computable
    increasing functions
    from $\mathbb N$ to $\mathbb N$ which $X$ would output if $X$ were commanded:
    \begin{quote}
        ``Until further notice, output (codes of) total computable increasing
        functions $h_1,h_2,\ldots$ from $\mathbb N$ to $\mathbb N$ satisfying the
        following constraints:
        \begin{enumerate}
            \item (Linear ordering) For any two $i,j\in\mathbb N$ with $i\neq j$, either
            $h_i\succ h_j$ or $h_j\succ h_i$ (Definition \ref{functionsuccdefn}).
            \item (Largeness) For every Turing machine $M$ such that you know
            $M$ computes a total computable function $h:\mathbb N\to\mathbb N$,
            there is some $i$ such that $h_i\succ h$.
            \item (Well-foundedness) There is no sequence $i_1,i_2,\ldots$ such that
            $h_{i_1}\succ h_{i_2}\succ\cdots$.
            \item (Pseudo-density)
            The $h_i$'s are to be as close as possible to being \emph{dense}---that is
            to say, they are to be as close as possible to having the property that
            for all $i,k\in\mathbb N$ with $h_k\succ h_i$, there is some $j\in\mathbb N$
            such that $h_k\succ h_j\succ h_i$---without violating the above
            well-foundedness constraint; you are to use your judgment and discretion
            to interpret this pseudo-density requirement.''
        \end{enumerate}
    \end{quote}
\end{definition}

The above pseudo-density constraint is intentionally vague, which we can get away
with because the definition does not depend on what pseudo-density actually means,
but only on how the AGI interprets the \emph{words} (i.e., how the AGI responds to a
specific well-defined stimulus, regardless of how vague the semantics of that stimulus
may be).
Thus, we are taking advantage of the AGI's ability to ``adapt with insufficient
knowledge and resources'' \cite{wang2019defining}. Note that the resulting $h_i$'s
cannot actually be dense, as that would violate well-foundedness: given any $i,j$
with $h_j\succ h_i$, there would be some $k_1$ such that $h_j\succ h_{k_1}\succ h_i$,
and then there would be some $k_2$ such that $h_{k_1}\succ h_{k_2}\succ h_i$,
and then there would be some $k_3$ such that $h_{k_2}\succ h_{k_3}\succ h_i$, and so
on, but then $h_{k_1}\succ h_{k_2}\succ \cdots$ would violate well-foundedness.
We will offer some motivation for pseudo-density below in
Remark \ref{pseudodensityremark}.

\begin{definition}
\label{alphaidefn}
    Let $X$ be an AGI. We define a sequence $\alpha^X_1,\alpha^X_2,\ldots$
    of ordinals by recursion as follows. Note that at first glance,
    the following recursive definition looks too circular to work.
    We will show in Lemma \ref{transfiniterecursionmagiclemma} that
    it avoids infinite regress and thus
    it does work. For each $i$, let $\alpha^X_i$ be the the smallest
    ordinal which is larger than every ordinal $\alpha^X_j$ such that
    $h^X_i\succ h^X_j$.
\end{definition}

\begin{lemma}
\label{transfiniterecursionmagiclemma}
    Definition \ref{alphaidefn} works (it does not lead to infinite regress).
\end{lemma}

\begin{proof}
    Assume, for the sake of contradiction, that Definition \ref{alphaidefn}
    leads to infinite regress. This means there is some $i_1$ such that
    in order to define $\alpha^X_{i_1}$ we must first define $\alpha^X_{i_2}$
    for some $i_2$, and in order to define $\alpha^X_{i_2}$ we must first
    define $\alpha^X_{i_3}$ for some $i_3$, and in order to define
    $\alpha^X_{i_3}$ we must first define $\alpha^X_{i_4}$ for some $i_4$,
    and so on forever. Thus, there is an infinite sequence $i_1,i_2,\ldots$
    such that in order to define each $\alpha^X_{i_j}$, we must first
    define $\alpha^X_{i_{j+1}}$. Now, each $\alpha^X_{i_j}$ is defined as
    the smallest ordinal larger than every ordinal $\alpha^X_j$ such that
    $h^X_{i_j}\succ h^X_j$. So, since defining $\alpha^X_{i_j}$ requires
    us to first define $\alpha^X_{i_{j+1}}$, this implies $i_{j+1}$ is
    such a $j$, i.e., that $h^X_{i_j}\succ h^X_{i_{j+1}}$.
    Thus $h^X_{i_1}\succ h^X_{i_2}\succ h^X_{i_3}\succ\cdots$,
    but this contradicts the \emph{well-foundedness} part of the command
    from Definition \ref{fairsequencedefn}.
\end{proof}

\begin{lemma}
\label{technicallemmaaboutalphaXi}
    Let $X$ be an AGI. For all $i,j$,
    $h^X_i\succ h^X_j$ if and only if $\alpha^X_i>\alpha^X_j$.
\end{lemma}

\begin{proof}
    Fix $i$ and $j$.
    By the \emph{linear ordering} part of the command in Definition
    \ref{fairsequencedefn}, either $h^X_i\succ h^X_j$ or $h^X_j\succ h^X_i$.
    Assume $h^X_i\succ h^X_j$, the other case is similar.
    By definition, $\alpha^X_i$ is defined to be the smallest ordinal
    which is larger than $\alpha^X_{j'}$ for every $j'$ such that
    $h^X_i\succ h^X_{j'}$. One such $j'$ is $j$ itself,
    so by definition $\alpha^X_i$ is larger than $\alpha^X_j$, as desired.
\end{proof}

\begin{definition}
\label{generalizedmajorizationhierarchydefn}
(Generalized majorization hierarchies)
    Let $X$ be an AGI. Let $\alpha=\sup_i \alpha^X_i$ be the smallest ordinal
    bigger than all of the $\alpha^X_i$'s. The \emph{generalized majorization
    hierarchy given by $X$} is the family $(g_\beta)_{\beta<\alpha}$ of functions
    labelled by ordinals below $\alpha$, where each $g_{\beta}$ is defined
    such that $g_{\beta}=h^X_i$ where $i$ is such that $\beta=\alpha^X_i$.
\end{definition}

\begin{remark}
\label{pseudodensityremark}
With Definition \ref{generalizedmajorizationhierarchydefn} in view,
the pseudo-density
constraint in Definition \ref{fairsequencedefn} can be better motivated:
the point of pseudo-density is to ensure that the ordinal
$\sup_i \alpha^X_i$ is as large as possible. An alternative to pseudo-density would
be to directly command (in Definition \ref{fairsequencedefn}) that the $h_i$'s should
be chosen in such a way as to make $\alpha=\sup_i\alpha^X_i$ large (leaving it up
to the AGI's judgment and discretion how to interpret that), but this would be
difficult because it would mean that Definition \ref{generalizedmajorizationhierarchydefn}
(and its dependencies) would need to be embedded into the command in
Definition \ref{fairsequencedefn}, which would require some significant acrobatics.
\end{remark}

A generalized majorization hierarchy (Definition \ref{generalizedmajorizationhierarchydefn})
serves as a partial solution to Problem \ref{bigoproblem}: given any AGI $X$, we can say
that a function $f:\mathbb N\to\mathbb N$ has growth rate
$\beta$ where $\beta$ is the smallest ordinal $<\sup_i \alpha^X_i$
such that $g_\beta\succ f$---or $f$ has growth rate $\infty$ if there is no
such $\beta$. To this solution to Problem \ref{bigoproblem}, there is a corresponding
Hibbard-style intelligence measure.

\begin{definition}
    (Transfinite Hibbard measures)
    Suppose $X$ is an AGI. By the \emph{transfinite Hibbard measure given by $X$},
    we mean the measure $\|\bullet\|_X$ which assigns to every AGI $Y$ an
    intelligence measure $\|Y\|_X$ defined as follows.
    $\|Y\|_X$ is defined to be the smallest $\beta<\sup_i\alpha^X_i$
    such that for all $\gamma<\beta$,
    $Y$'s predictor learns every evader $e\in E_\gamma$ (or $\|Y\|_X=\infty$ if
    $Y$'s predictor learns every evader $e\in E_\gamma$ for all
    $\gamma<\sup_i \alpha^X_i$).
\end{definition}


\section{Do all AGIs have original Hibbard intelligence $\infty$?}
\label{trivialitysection}

In this section, we will argue that under certain idealizing
assumptions, every AGI should have original Hibbard intelligence $\infty$.
Thus, we would suggest that Definition \ref{classichibbardmeasuredefn} is really
only appropriate for certain weak AIs who are strong enough that they can be
induced to participate in adversarial sequence prediction games, but who are
too weak to qualify as genuine AGIs.


\subsection{A semi-optimal predictor}
\label{semioptimalpredictorsubsection}

The most obvious way to try to predict an evader would be to
enumerate some list $M_1,M_2,\ldots$ of Turing machines and
sequentially try to predict $e$ by assuming $e$ computes the same
function as $M_1$, and if that ever fails, then predict $e$ by
assuming $e$ computes the same function as $M_2$, and if that ever
fails, then predict $e$ by assuming $e$ computes the same function
as $M_3$, and so on. In doing so, we should limit ourselves
to Turing machines $M_i$ which define total functions, lest
we end up waiting forever for a non-terminating computation
and never getting around to making our next
prediction. Unfortunately, the set of total Turing machines is
not computably enumerable, so the best we can hope for is to enumerate
some subset of the total Turing machines. But which subset? This is
a tricky and arbitrary decision---thankfully, we do not need to
decide it ourselves, we can let the AGI decide for us.

\begin{definition}
\label{bruteforcepredictordefn}
    (Brute-Force Predictors)
    Suppose $X$ is an AGI. Let $M_1,M_2,\ldots$ be the list of Turing machines
    which $X$ would enumerate if $X$ were commanded: ``Enumerate all the Turing
    machines which you know define total functions from $B^*$ to $B$''.
    By \emph{$X$'s brute-force predictor}, we mean the predictor $p$ defined
    as follows.
    \begin{enumerate}
        \item
        Initially, $p$ shall attempt to predict the evader $e$ by assuming that $e$
        defines the same function as $M_1$. When (if ever) this fails,
        say that \emph{$M_1$ is ruled out from being the evader}.
        \item
        Once $M_1$ has been ruled out from being the evader (if ever),
        $p$ shall attempt to predict $e$ by assuming that $e$ defines the same
        function as $M_2$. When (if ever) this fails,
        say that \emph{$M_2$ is ruled out from being the evader}.
        \item
        Once $M_2$ has been ruled out from being the evader (if ever),
        $p$ shall attempt to predict $e$ by assuming that $e$ defines the same
        function as $M_3$. When (if ever) this fails,
        say that \emph{$M_3$ is ruled out from being the evader}.
        \item
        And so on forever...
    \end{enumerate}
\end{definition}

Note that $X$'s brute-force predictor
is total precisely because of the mathematical truthfulness assumption
(from Definition \ref{idealizingassumption}): whenever $X$ knows that
$M_i$ defines a total function from $B^*$ to $B$, $M_i$ really \emph{does}
define a total function from $B^*$ to $B$ (hereafter, we will suppress
remarks like this and routinely use the mathematical truthfulness assumption
without explicit mention).

\begin{lemma}
\label{knowingimplieslearninglemma}
    Let $X$ be an AGI and let $e$ be an evader (so $e$ is a Turing machine which
    computes a function from $B^*$ to $B$).
    Let $M$ be any Turing machine which computes the same function from $B^*$ to $B$
    as $e$ computes.
    If $X$ knows that $M$ defines a total function from $B^*$ to $B$,
    then $X$'s brute-force predictor learns to predict $e$.
\end{lemma}

\begin{proof}
    Let $p$ be $X$'s brute-force predictor.
    Let $M_1,M_2,\ldots$ be as in Definition \ref{bruteforcepredictordefn}.
    Since $X$ knows that $M$ defines a total
    function from $B^*$ to $B$,
    it follows that $M=M_k$ for some $k$.
    When $p$ plays against $e$, it cannot occur that $M_k$ is ruled out
    from being the evader, because in order for that to occur, $p$ would have
    to fail at predicting $e$ when $p$ assumes that $e$ computes the same
    function as $M_k$, but that cannot fail because that assumption is true.
    Since $M_k$ cannot be ruled out from being the evader, it follows that
    step $k+1$ of Definition \ref{bruteforcepredictordefn} will never be
    reached, which in turn implies that $p$ stops failing at predicting $e$
    after finitely many initial failures, in other words, $p$ learns to
    predict $e$.
\end{proof}

For any AGI $X$, we would like to claim that $X$'s brute-force predictor is
optimal among all predictors $X$ could act as; unfortunately this
is not true in the strongest sense it possibly could be true, as the following
example shows.

\begin{example}
\label{bruteforcenottotallyoptimalexample}
    Let $X$ be an AGI. Let $f:\mathbb N\to B$ be a total computable function
    which is not computed\footnote{Such an $f$ must exist because
    otherwise, $X$ could enumerate all the total
    computable functions from $\mathbb N$ to $B$, which is absurd because those functions
    are not computably enumerable.} by any Turing machine $M$ such that $X$ knows $X$ computes
    a total function from $\mathbb N$ to $B$.
    For each $i\in\{0,1\}$, let $e_i$ be an evader such that:
    \begin{align*}
        e_i(\langle\rangle) &= i,\\
        e_i(y_1,\ldots,y_n) &=
        \begin{cases}
            i &\mbox{if $y_1=\cdots=y_n=i$,}\\
            f(n) &\mbox{otherwise.}
        \end{cases}
    \end{align*}
    For each $i\in\{0,1\}$, let $p_i$ be a
    constantly-$i$ predictor, $p_i(x_1,\ldots,x_n)=i$.
    Then $p_0$ learns to predict $e_0$ and $p_1$ learns to predict $e_1$,
    but $X$'s brute-force predictor learns to predict at most one of $e_0$ or $e_1$.
\end{example}

\begin{proof}
    Clearly the result of $p_0$ playing against $e_0$ is
    $(x_0,y_0,x_1,y_1,\ldots)=(0,0,0,0,\ldots)$, so $p_0$ learns to predict $e_0$.
    Likewise, the result of $p_1$ playing against $e_1$ is
    $(x_0,y_0,x_1,y_1,\ldots)=(1,1,1,1,\ldots)$, so $p_1$ learns to predict $e_1$.
    It remains to show that $X$'s brute-force predictor $p$ cannot learn to predict
    both $e_0$ and $e_1$.

    Let $M_1,M_2,\ldots$ be as in Definition \ref{bruteforcepredictordefn}.
    Let $g:B^*\to B$ be the function computed by $M_1$.

    Case 1: $g(\langle\rangle)=1$. Then when $p$ plays against $e_0$,
    after the first step, it will never be the case that
    $y_1=\cdots=y_n=0$ (because $y_1=g(\langle\rangle)=1$).
    Thus, for all $n>0$, $e_0(y_1,\ldots,y_n)=f(n)$.
    I claim that $p$ does not learn $e_0$.
    To see this, assume (for the sake of contradiction)
    that $p$ learns $e_0$. It follows that there is some $k$ (which we may take
    as small as possible) such that $M_k$ never gets ruled out from being the
    evader (Definition \ref{bruteforcepredictordefn}). Let $h:B^*\to B$ be the
    function defined by $M_k$.
    Let $(x_0,y_0,x_1,y_1,\ldots)$ be the result of $p$ playing against $e_0$.
    It follows that there is some $j$ such that for all $i>j$,
    $h(y_1,\ldots,y_{i-1})=x_i=y_i=f(i)$.
    This shows that for all but finitely many $i\in\mathbb N$,
    $f(i)=h(y_1,\ldots,y_{i-1})$. Since $X$ knows that $M_k$ defines a total
    function from $B^*$ to $B$, it follows that
    there is a Turing machine $M$ such that $M$ computes $f$ and $X$ knows
    $M$ computes a total function from $\mathbb N$ to $B$. Absurd, this
    contradicts our choice of $f$.

    Case 2: $g(\langle\rangle)=0$. Then by similar reasoning as in Case 1,
    it can be shown that $p$ does not learn $e_1$.
\end{proof}

Example \ref{bruteforcenottotallyoptimalexample} shows that we cannot
hope for the brute-force to be totally optimal in the most extreme possible sense:
there will always be evaders that the AGI's brute-force predictor
fails to learn, but which other
predictors (which the AGI is capable of acting as)
do nevertheless learn. Example \ref{bruteforcenottotallyoptimalexample} involves
highly contrived evaders which are custom-made to act stupidly in one specific case
(allowing them to be learned by stupid predictors), while being
highly sophisticated in other cases. In the following definition, we rule out
situations where a less sophisticated predictor manages to learn a
more sophisticated evader due to the evader concealing its true sophistication from
the predictor.

\begin{definition}
\label{notwithoutafightdefn}
    Suppose $p$ is a predictor and $e$ is an evader.
    We say that \emph{$p$ learns $e$ but-not-without-a-fight}
    if the following conditions hold:
    \begin{enumerate}
        \item $p$ learns $e$.
        \item For each $i$, $t_p(i)\geq t_e(i)$.
    \end{enumerate}
\end{definition}

Armed with Definition \ref{notwithoutafightdefn}, we can now state a theorem
which shows the semi-optimality of the brute-force predictor.

\begin{theorem}
\label{semioptimalitytheorem}
    (Semi-optimality of brute force)
    Let $X$ be an AGI and let $p$ be any predictor.
    If $X$ knows that $p$ computes a total function from $B^*$ to $B$,
    and if $p$ learns an evader $e$ but-not-without-a-fight,
    then $X$'s brute-force predictor learns $e$.
\end{theorem}

\begin{proof}
    Since $X$ knows $p$ computes a total
    function from $B^*$ to $B$, $X$ knows that $e'$ is a total
    computable function from $B^*$ to $B$, where $e'$ is the Turing machine which
    takes an input $b\in B^*$ and operates as follows:
    \begin{enumerate}
        \item
        Calculate $t_p(i)$ (by running $p$ on all length-$i$ binary sequences).
        \item
        Run $e$ on $b$ for up to $t_p(i)$ steps. If $e$ outputs a result $x$ within that
        time, then output $x$. Otherwise, output $0$.
    \end{enumerate}
    Since $p$ learns $e$ but-not-without-a-fight, each $t_p(i)\geq t_e(i)$,
    so in fact $e'$ computes the same function as $e$.
    By Lemma \ref{knowingimplieslearninglemma}, $X$'s brute-force predictor
    learns $e'$. Since $e'$ and $e$ compute the same function, this implies
    $X$'s brute-force predictor learns $e$.
\end{proof}

\subsection{Triviality of the original Hibbard measure}
\label{trivialitysubsection}

Although Example \ref{bruteforcenottotallyoptimalexample} showed that
an AGI $X$'s brute-force predictor
is not optimal in the strongest possible sense, Theorem \ref{semioptimalitytheorem}
shows that $X$'s brute-force predictor is still semi-optimal. In some sense,
$X$'s brute-force predictor learns every evader which any other predictor $p$ (that $X$
knows is a predictor) would learn, except possibly for cases where $p$ only learns
an evader $e$ because $e$ conceals its full sophistication from $p$.

Since we only used basic mathematics to prove Theorem \ref{semioptimalitytheorem},
any genuine AGI should also eventually figure out Theorem \ref{semioptimalitytheorem}.
Thus, when an AGI is commanded to compete as a predictor in the adversarial
sequence prediction game, the resulting predictor which the AGI acts as should be
at least as good as the brute-force predictor, because why would the AGI act as
a worse predictor if it can figure out that brute-force is semi-optimal?
We formalize this in the following assumption.

\begin{assumption}
\label{bruteforceassumption}
    For every AGI $X$, $X$'s predictor is at least as good as
    $X$'s brute-force predictor, by which we mean that for every evader $e$,
    if $X$'s brute-force predictor learns $e$, then $X$'s predictor
    learns $e$.
\end{assumption}

Based on Assumption \ref{bruteforceassumption},
we are now ready to argue that the original Hibbard measure is
trivial: that it assigns intelligence level $\infty$ to every AGI.

\begin{theorem}
    Every AGI $X$ has original Hibbard intelligence $\infty$.
\end{theorem}

\begin{proof}
    Let $p$ be $X$'s brute-force predictor, and let $q$ be $X$'s predictor.
    By Definition \ref{generalintelligencemeasuredefn}, in order
    to show $X$ has original Hibbard intelligence $\infty$, we must
    show that $q$ learns to predict $e$ for every $e$ in $E_{f_m}$ for
    every $m\in\mathbb N$,
    where each $f_m:\mathbb N\to\mathbb N$ is defined by
    $f_m(k)=\max_{0<i\leq m}\max_{j\leq k}g_i(j)$,
    where $g_i$ is the $i$th primitive recursive function
    in Liu's enumeration
    (Definition \ref{classichibbardmeasuredefn}).
    By Assumption \ref{bruteforceassumption}, it suffices to show that
    $q$ learns all such $e$, for then Assumption \ref{bruteforceassumption}
    says that $p$ learns all such $e$ as well.

    Let $m\in\mathbb N$ be arbitrary and let $e\in E_{f_m}$ be arbitrary.
    By Definition \ref{evadersetdefinition},
    for all but finitely many $n\in\mathbb N$, $t_e(n)<f_m(n)$.
    We will show that $p$ learns to predict $e$.

    Let $M$ be a Turing machine which takes an input
    $(y_1,\ldots,y_n)\in B^*$ and
    operates as follows:
    \begin{itemize}
        \item
        Spend up to $f_m(n)$ steps computing $e(y_1,\ldots,y_n)$.
        If $e$ halts (with output $x$) during that time, then output $x$.
        Otherwise, output $0$.
    \end{itemize}
    Since, for all but finitely many $n\in\mathbb N$, $t_e(n)<f_m(n)$,
    it follows that $M$ computes the same function as $e$ except for finitely
    many exceptions.

    It is well-known that for every primitive recursive function $g$, Peano
    arithmetic proves that $g$ is total. Thus, by the \emph{basic mathematical
    exposure} and \emph{mathematical reasoning ability} assumptions from
    Assumption \ref{idealizingassumption}, it follows that $X$ knows that
    $M$ is total. Since $M$ computes the same function as $e$ except for
    finitely many exceptions, it follows that there is a Turing machine $M'$
    which computes the same function as $e$, and such that $X$
    knows that $M'$ is total. By Lemma \ref{knowingimplieslearninglemma},
    $p$ learns to predict $e$.
\end{proof}

We will finish this section by stating one other important, possibly
counterintuitive consequence of Assumption \ref{bruteforceassumption}.

\begin{theorem}
\label{counterintuitivetheorem}
    If $X$ is an AGI and $p$ is $X$'s predictor, then $X$ does not know that
    $p$ defines a total function from $B^*$ to $B$.
\end{theorem}

\begin{proof}
    Let $e$ be the evader which attempts to evade the predictor by assuming that
    the predictor is $p$ and always playing the opposite digit which $p$ is
    about to play. Clearly $p$ does not learn $e$.

    The fact that $e$ defines a total function from $B^*$ to $B$ follows
    by basic mathematical reasoning from the fact that $p$ defines a total
    function from $B^*$ to $B$. Thus, if $X$ knows that $p$ defines a total
    function from $B^*$ to $B$, then $X$ should know that $e$ defines a total
    function from $B^*$ to $B$. If so, then $X$'s brute-force predictor learns
    $e$ by Lemma \ref{knowingimplieslearninglemma}. But then
    Assumption \ref{bruteforceassumption} would imply that $p$ learns $e$,
    a contradiction.
\end{proof}

Theorem \ref{counterintuitivetheorem} is reminiscent of
Examples \ref{counterintuitiveexample1} and \ref{counterintuitiveexample2}.



\section{Summary and Conclusion}
\label{conclusionsection}

To summarize:
\begin{itemize}
    \item
    Hibbard proposed \cite{hibbard} an intelligence measure for predictors (and, implicitly,
    for AGIs capable of acting as predictors) in games of adversarial sequence prediction.
    \item
    Hibbard's measure depends on one specific sequence of computable functions
    from $\mathbb N$ to $\mathbb N$.
    \item
    In Section \ref{agiperspectivesection}, we argued that all AGIs probably have
    intelligence $\infty$ according to Hibbard's definition. Essentially, this is because
    the specific functions that Hibbard's definition is based on should all be trivially
    well-known by any genuine AGI.
    \item
    In Section ..., we introduced what we call \emph{simple Hibbard
    measures}. For each AGI $X$, there is a corresponding relative Hibbard measure which
    assigns an intelligence measurement $|Y|_X$ to every AGI $Y$. The relative Hibbard
    measure corresponding to $X$ might be thought of as a measure of ``intelligence as
    judged by $X$''. The key insight needed to define the
    relative Hibbard measure corresponding
    to $X$ is as follows: rather than depend on one fixed sequence of functions (as in
    Hibbard's original definition), we can delegate the task of choosing a function-sequence
    to $X$.
    \item
    We showed that relative Hibbard measures are less trivial than Hibbard's original measure.
    However, they still suffer a crucial flaw: the Hibbard measure corresponding to an
    AGI $X$ depends unnaturally on the \emph{order} in which $X$ can think of
    computable functions.
    \item
    To remedy the above-mentioned flaw in the relative Hibbard measures, we reviewed
    \emph{slow-growing hierarchies} from mathematical logic and used them to define
    (in Section ...) what
    we call \emph{transfinite Hibbard measures}, which lead to
    computable-ordinal-number-valued
    intelligence measures rather than real-number-valued intelligence measures.
\end{itemize}

This paper exemplifies a useful technique for theoretical AGI research.
By viewing an AGI as an employee who is capable of engaging in casual human
language communication with its employer, and which can be commanded (using said
casual human language) to perform mathematical tasks, the AGI becomes a useful
black box to which many arbitrary decisions can be delegated. For example, instead
of arbitrarily choosing one particular function-sequence (as Hibbard did in order
to define his original measure), we can short-circuit the task of
function-sequence-selection by delegating it to an AGI.


\bibliographystyle{plain}
\bibliography{hibbard}
\end{document}
