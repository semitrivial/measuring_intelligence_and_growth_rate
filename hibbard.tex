\documentclass[twoside,11pt]{article}
\usepackage{jagi}
\usepackage[utf8]{inputenc}
\usepackage{natbib}
\usepackage{amssymb}
\usepackage{amsmath}
\usepackage{mathrsfs}
\usepackage{mathdots}

\jagiheading{TBD}{TBD}{TBD}{TBD}{TBD}{TBD}{TBD}{S.\ Alexander \& B.\ Hibbard}

\ShortHeadings{Measuring intelligence and growth rate}{S.\ Alexander \& B.\ Hibbard}

\title{Measuring intelligence and growth rate: variations on
Hibbard's intelligence measure}

\author{\name Samuel Alexander \\
    \email samuelallenalexander@gmail.com \\
    \addr (Primary Author)\\
    Quantitative Research Analyst\\
    The U.S.\ Securities and Exchange Commission\\
    New York Regional Office\\
    \AND
    \name Bill Hibbard \\
    \addr Space Science and Engineering Center \\
    University of Wisconsin \\
    Madison, WI
}

\editor{TBD}

\begin{document}

\maketitle

\begin{abstract}
    In 2011, Hibbard suggested an intelligence measure for agents
    who compete in an adversarial sequence prediction game. We argue
    that Hibbard's idea should actually be considered as two separate
    ideas: first, that the intelligence of such agents can be measured
    based on the growth rates of the runtimes of the competitors that
    they defeat; and second, one specific (somewhat arbitrary) method for measuring said
    growth rates. Whereas Hibbard's intelligence measure is based on the latter
    growth-rate-measuring method, we survey
    other methods for measuring function
    growth rates, and exhibit the resulting Hibbard-like intelligence measures
    and taxonomies. Of particular interest, we obtain intelligence taxonomies
    based on Big-O and Big-Theta notation systems, which taxonomies
    are novel in that they challenge conventional notions of what an
    intelligence measure should look like. We discuss how intelligence measurement
    of sequence predictors can indirectly serve as intelligence measurement for
    agents with Artificial General Intelligence (AGIs).
\end{abstract}

\section{Introduction}

In his insightful paper, \citet{hibbard} introduces a novel
intelligence measure (which we will here refer to as the \emph{original Hibbard measure})
for agents who play a game of adversarial sequence prediction
\citep{hibbard2008adversarial}
``against a hierarchy of increasingly difficult sets of'' evaders (environments that attempt
to emit $1$s and $0$s in such a way as to evade prediction).
The levels of Hibbard's hierarchy are labelled by natural numbers, and
an agent's original Hibbard measure is the maximum $n\in\mathbb N$ such that
said agent learns to predict all the evaders in the $n$th level of the hierarchy,
or implicitly\footnote{Hibbard does not explicitly include the $\infty$ case in his
definition, but in his Proposition 3 he refers to agents having ``finite intelligence'', and
it is clear from context that by this he means agents who fail to predict some evader
somewhere in the hierarchy.} an agent's original Hibbard measure is $\infty$
if said agent learns to predict all the evaders in all levels of Hibbard's hierarchy.

The hierarchy which Hibbard uses to measure intelligence is based on the growth
rates of the runtimes of evaders.
We will argue that Hibbard's idea is really a combination of two
orthogonal ideas. First: that in some sense the intelligence of a predicting agent
can be measured based on the growth rates of the runtimes of the evaders whom that
predictor learns to predict. Second: Hibbard proposed one particular method for
measuring said growth rates. The growth rate measurement which Hibbard proposed yields
a corresponding intelligence measure for these agents. We will argue that \emph{any}
method for measuring growth rates of functions yields a corresponding
\emph{adversarial sequence prediction intelligence} measure (or \emph{ASPI} measure
for short) provided the underlying number system provides a way of choosing canonical bounds
for bounded sets. If the underlying number system does not provide a way of choosing
canonical bounds for bounded sets, the growth-rate-measure will yield a corresponding
ASPI taxonomy (like the big-$O$ taxonomy of asymptotic complexity).

The particular method which Hibbard used to measure function growth rates is
not very standard. We will survey other
ways of measuring function growth rates,
and these will yield corresponding ASPI measures and taxonomies.

The structure of the paper is as follows.
\begin{itemize}
    \item
    In Section \ref{originalmeasuresection}, we review the original Hibbard measure.
    \item
    In Section \ref{growthratesection}, we argue that any method of measuring
    the growth rate of functions yields an ASPI measure or taxonomy,
    and that the original Hibbard measure is just a special case resulting from
    one particular method of measuring function growth rate.
    \item
    In Section \ref{bigosection}, we consider Big-O notation and Big-$\Theta$ notation
    and define corresponding ASPI taxonomies.
    \item
    In Section \ref{exoticsection}, we consider several numeric solutions to
    the problem of measuring the growth rate of functions (using various number
    systems), and define
    corresponding ASPI measures and taxonomies.
    \item
    In Section \ref{prosandconssection}, we give pros and cons of different
    ASPI measures and taxonomies.
    \item
    In Section \ref{conclusionsection}, we summarize and make concluding remarks.
\end{itemize}

\section{Hibbard's original measure}
\label{originalmeasuresection}

Hibbard proposed an intelligence measure for measuring the intelligence
of agents who compete to predict evaders in a game of
adversarial sequence prediction (we define this
formally below). A predictor $p$ (whose intelligence we want to measure)
competes against evaders $e$. In each step of the game,
both predictor and evader simultaneously choose a binary digit, $1$ or $0$.
Only after both of them have made their choice do they see which choice the other
one made, and then the game proceeds to the next step. The predictor's goal in
each round is to choose the same digit that the evader will choose;
the evader's goal is to choose a different digit than the predictor. The predictor
wins the game (and is said to \emph{learn to predict $e$}, or simply to
\emph{learn $e$}) if, after finitely many
initial steps, eventually the predictor always chooses the same digit as the
evader.

\begin{definition}
By $B$, we mean the binary alphabet $\{0,1\}$. By $B^*$, we mean the set of all
finite binary sequences. By $\langle\rangle$ we mean the empty binary sequence.
\end{definition}

\begin{definition}
\label{evaderpredictordefn}
    (Predictors and evaders)
    \begin{enumerate}
        \item
        By a \emph{predictor}, we mean a Turing machine $p$
        which takes as input a finite (possibly empty) binary sequence
        $(x_1,\ldots,x_n)\in B^*$
        (thought of as a sequence of \emph{evasions})
        and outputs $0$ or $1$ (thought of as a \emph{prediction}), which output
        we write as $p(x_1,\ldots,x_n)$.
        \item
        By an \emph{evader}, we mean a Turing machine $e$
        which takes as input a finite (possibly empty) binary sequence
        $(y_1,\ldots,y_n)\in B^*$
        (thought of as a sequence of \emph{predictions})
        and outputs $0$ or $1$ (thought of as an \emph{evasion}), which output
        we write as $e(y_1,\ldots,y_n)$.
        \item
        For any predictor $p$ and evader $e$, the \emph{result of $p$ playing the
        game of adversarial sequence
        prediction against $e$} (or more simply, the \emph{result of
        $p$ playing against $e$}) is the infinite binary sequence
        $(x_1,y_1,x_2,y_2,\ldots)$
        defined as follows:
        \begin{enumerate}
            \item
            The first evasion
            $x_1=e(\langle\rangle)$ is
            the output of $e$ when run on the empty prediction-sequence.
            \item
            The first prediction
            $y_1=p(\langle\rangle)$ is
            the output of $p$ when run on the empty evasion-sequence.
            \item
            For all $n>0$, the $(n+1)$th evasion
            $x_{n+1}=e(y_1,\ldots,y_n)$ is
            the output of $e$ on the sequence of the first $n$ predictions.
            \item
            For all $n>0$, the $(n+1)$th prediction
            $y_{n+1}=p(x_1,\ldots,x_n)$ is
            the output of $p$ on the sequence of the first $n$ evasions.
        \end{enumerate}
        \item
        Suppose $r=(x_1,y_1,x_2,y_2,\ldots)$ is the result of a predictor $p$ playing
        against an evader $e$. For every $n\geq 1$,
        we say \emph{the predictor wins round $n$ in $r$}
        if $x_n=y_n$; otherwise,
        \emph{the evader wins round $n$ in $r$}.
        We say that \emph{$p$ learns to predict $e$}
        (or simply that \emph{$p$ learns $e$}) if there is some $N\in\mathbb N$
        such that for all $n>N$, $p$ is the winner of round $n$ in $r$.
    \end{enumerate}
\end{definition}

Note that if $e$ simply ignores its inputs $(y_1,\ldots,y_n)$ and instead
computes $e(y_1,\ldots,y_n)$ based only on $n$, then $e$ is essentially a sequence.
Thus Definition \ref{evaderpredictordefn} is a generalization of sequence prediction,
which many authors have written about (such as \citet{legg2006there}, who gives many
references).

In the following definition, we differ from Hibbard's original paper
because of a minor (and fortunately, easy-to-fix) error there.

\begin{definition}
\label{tsubedefinition}
    Suppose $e$ is an evader.
    For each $n\in\mathbb N$, let $t_e(n)$ be the maximum number of steps that $e$ takes
    to run on any length-$n$ sequence of binary digits.
    In other words, $t_e(0)$ is the number of steps $e$ takes to run on $\langle\rangle$,
    and for all $n>0$,
    \[
        t_e(n) = \max_{b_1,\ldots,b_n\in \{0,1\}}
        (\text{number of steps $e$ takes to run on $(b_1,\ldots,b_n)$}).
    \]
\end{definition}

\begin{example}
    Let $e$ be an evader. Then
    $t_e(2)$ is equal to the number of steps $e$ takes to run on input
    $(0,0)$, or to run on input $(0,1)$, or to run on input $(1,0)$, or to run on input
    $(1,1)$---whichever of these four possibilities is largest.
\end{example}

\begin{definition}
\label{functionsuccdefn}
    Suppose $f:\mathbb N\to\mathbb N$ and $g:\mathbb N\to\mathbb N$.
    We say $f$ \emph{majorizes} $g$, written
    $f\succ g$, if there is some $n_0\in \mathbb N$ such that for all
    $n>n_0$, $f(n)>g(n)$.
\end{definition}

\begin{definition}
\label{evadersetdefinition}
    Suppose $f:\mathbb N\to\mathbb N$. We define
    $E_f$ to be the set of all evaders $e$ such that
    $f\succ t_e$.
\end{definition}

\begin{definition}
\label{classichibbardmeasuredefn}
    (The original Hibbard measure)
    Let $g_1,g_2,\ldots$ be the enumeration of the primitive recursive
    functions given by \citet{liu1960enumeration}.
    For each $m>0$, define $f_m:\mathbb N\to\mathbb N$ by
    \[f_m(k)=\max_{0<i\leq m}\max_{j\leq k}g_i(j).\]
    For any predictor $p$, we define the \emph{original Hibbard intelligence of $p$}
    to be the maximum $m>0$
    such that $p$ learns to predict $e$ for every $e\in E_{f_m}$
    (or $0$ if there is no such $m$, or $\infty$ if $p$ learns to predict $e$
    for every $e\in E_{f_m}$ for every $m>0$).
\end{definition}

\subsection{Predictor intelligence and AGI intelligence}
\label{agiproxysection}

Definition \ref{classichibbardmeasuredefn}, and similar measures and taxonomies
which we will
define later, quantify the intelligence of predictors in the game of
adversarial sequence prediction.
But any method for quantifying the intelligence of such predictors can also
approximately quantify the intelligence of (suitably idealized)
agents with Artificial General Intellience (that is, the intelligence of AGIs).

Presumably, a suitably idealized AGI should be capable of understanding, and
obedient in following or trying to follow, commands issued in everyday human
language\footnote{It is somewhat unclear how explicitly an AGI would
obey certain commands. To use an example of
\citet{yampolskiycontrol}, if we asked a car-driving AGI
to stop the car, would the AGI stop the car
in the middle of traffic, or would it pull over to the side first?
We assume this ambiguity does not apply when we ask the AGI to perform
tasks of a sufficiently abstract and mathematical nature.}.
For example, if an AGI were commanded, ``until further notice, compute and list the
digits of pi,'' the AGI should be capable of understanding that command, and should
obediently compute said digits until commanded otherwise\footnote{Our thinking
here is reminiscent of some remarks of \citet{yampolskiy2013turing}.}.

It is unclear how an AGI ought to respond if given an impossible command,
such as ``write a computer program
that solves the halting problem'', or Yampolskiy's
``Disobey!'' \citep{yampolskiycontrol}. But an AGI should be capable of
understanding and attempting to obey an open-ended command, provided it is not
impossible. For example, we could command an AGI to ``until further notice,
write an endless poem about trees,'' and the AGI should be able to do so, writing
said poem line-by-line until we tell it to stop. This is despite the fact that the
command is open-ended and under-determined
(there are many decisions involved in writing a
poem about trees, and we have left all these decisions to the AGI's discretion).
The AGI's ability to obey such open-ended and under-determined commands
exemplifies its
ability to ``adapt with insufficient
knowledge and resources'' \citep{wang2019defining}.
One well-known example of an open-ended command which an AGI should be perfectly
capable of attempting to obey (perhaps at peril to us) is
Bostrom's ``manufacture as many paperclips as possible'' \citep{bostrom2003ethical}.

In particular, an AGI $X$ should be perfectly capable of obeying the following command:
``Act as a predictor in the game of adversarial sequence prediction''.
By giving $X$ this command, and then immediately filtering out all $X$'s
sensory input except only for input about the digits chosen by an evader,
we would obtain a formal predictor in the sense of Definition \ref{evaderpredictordefn}.
This predictor might be called ``the predictor generated by $X$''. Strictly speaking,
if the command is given to $X$ at time $t$, then it would be more proper to call
the resulting predictor ``the predictor generated by $X$ at time $t$'': up until
time $t$, the observations $X$ makes about the universe might have an effect on
the strategy $X$ chooses to take once commanded to act as a predictor; but as long
as we filter $X$'s sensory input immediately after giving $X$ the command, no
further such observations can so alter $X$'s strategy.
In short, to use Yampolskiy's terminology \citep{yampolskiy2012ai}, the act of
trying to predict adversarial sequence evaders is \emph{AI-easy}.

Thus, any intelligence measure (or taxonomy) for predictors also serves as an intelligence
measure (or taxonomy) for AGIs. Namely: the intelligence level of an AGI $X$ is equal to the
intelligence level of $X$'s predictor. Of course, a priori,
$X$ might be very intelligent at various other things while being poor
at sequence prediction, or vice versa, so this only
approximately captures $X$'s true intelligence.


\section{Quantifying growth rates of functions}
\label{growthratesection}

The following is a general and open-ended problem.

\begin{problem}
\label{bigoproblem}
    Quantify the growth-rate of functions from $\mathbb N$ to $\mathbb N$.
\end{problem}

The definition of the original Hibbard measure
(Definition \ref{classichibbardmeasuredefn})
can be thought of as implicitly depending on a specific solution to Problem
\ref{bigoproblem}, which we make explicit in the following definition.

\begin{definition}
\label{hibbardgrowthratedefn}
    For each $m>0$, let $f_m$ be as in Definition \ref{classichibbardmeasuredefn}.
    For each $f:\mathbb N\to\mathbb N$, we define the \emph{original Hibbard growth rate}
    $H(f)$ to be $\min\{m>0\,:\,f_m\succ f\}$ if there is any such $m>0$, and otherwise
    $H(f)=\infty$.
\end{definition}

\begin{lemma}
\label{straightfwdtechnicallemma}
    For every natural $m>0$ and every $f:\mathbb N\to\mathbb N$,
    $H(f)\leq m$ if and only if $f_m\succ f$.
\end{lemma}

\begin{proof}
    Straightforward.
\end{proof}

\begin{definition}
\label{variationondefinitionofEdefn}
    For every $m\in\mathbb N$, let $E^H_m$
    be the set of all evaders $e$ such that $H(t_e)\leq m$.
\end{definition}

\begin{lemma}
\label{equivalenceoftwoevadersetslemma}
    For every natural $m>0$, $E^H_m=E_{f_m}$.
\end{lemma}

\begin{proof}
    Let $e$ be an evader. By Definition \ref{variationondefinitionofEdefn},
    $e\in E^H_m$ if and only if $H(t_e)\leq m$.
    By Lemma \ref{straightfwdtechnicallemma}, $H(t_e)\leq m$ if and only if
    $f_m\succ t_e$. But by Definition \ref{evadersetdefinition}, this is the
    case if and only if $e\in E_{f_m}$.
\end{proof}

\begin{corollary}
\label{rephrasinghibbardsmeasurecorollary}
    For every predictor $p$, the original Hibbard measure of $p$
    is equal to the maximum natural $m>0$ such that
    $p$ learns $e$ whenever $e\in E^H_m$,
    or is equal to $0$ if there is no such $m$,
    or is equal to $\infty$
    if $p$ learns $e$ whenever $e\in E^H_m$ for all $m>0$.
\end{corollary}

\begin{proof}
    Immediate by Lemma \ref{equivalenceoftwoevadersetslemma}
    and Definition \ref{classichibbardmeasuredefn}.
\end{proof}

In other words, if $S$ is the set of all the $m$ as in Corollary
\ref{rephrasinghibbardsmeasurecorollary}, then the original Hibbard
measure of $p$ is the ``canonical upper bound'' of $S$, where
by the ``canonical upper bound'' of a set of natural numbers
we mean the maximum element of that set (or $\infty$ if that set
is unbounded).

\begin{remark}
\label{epiphanyremark}
Corollary \ref{rephrasinghibbardsmeasurecorollary} shows that
the definition of the original Hibbard measure can be rephrased in such a way
as to show that it depends in a uniform way on a particular solution to
Problem \ref{bigoproblem}, namely on the solution proposed by
Definition \ref{hibbardgrowthratedefn}. For \emph{any} solution $H'$ to
Problem \ref{bigoproblem}, we could define evader-sets $E^{H'}_m$ in a similar
way to Definition \ref{variationondefinitionofEdefn}, and, by copying
Corollary \ref{rephrasinghibbardsmeasurecorollary}, we could obtain a corresponding
intelligence measure given by $H'$
(provided there be some way of choosing canonical bounds of bounded sets in the
underlying number system---if not, we would have to be content with a taxonomy
rather than a measure, a predictor's intelligence falling into many nested
taxa corresponding to many different upper bounds, just as in Big-O notation a
function can simultaneously be $O(n^2)$ and $O(n^3)$).
This formalizes what we claimed in the Introduction,
that Hibbard's idea can be decomposed into two sub-ideas, firstly, that a predictor's
intelligence can be classified in terms of the growth rates of the runtimes of the
evaders it learns, and secondly, a particular method
(Definition \ref{hibbardgrowthratedefn})
of measuring those growth rates (i.e., a particular solution to
Problem \ref{bigoproblem}).
\end{remark}


\section{Big-O and Big-$\Theta$ intelligence}
\label{bigosection}

One of the most standard solutions
to Problem \ref{bigoproblem} in computer science is to categorize
growth rates of arbitrary functions by comparing them to more familiar functions using
Big-O notation or Big-$\Theta$ notation.
\citet{knuth1976big} defines these as follows
(we modify the definition slightly because
we are only concerned here with functions from $\mathbb N$ to $\mathbb N$).

\begin{definition}
\label{bigodefn}
    Suppose $f:\mathbb N\to\mathbb N$. We define the following function-sets.
    \begin{itemize}
        \item
        $O(f(n))$ is the set of all $g:\mathbb N\to\mathbb N$ such that
        there is some real $C>0$ and some $n_0\in\mathbb N$ such that
        for all $n\geq n_0$, $g(n)\leq Cf(n)$.
        \item
        $\Theta(f(n))$ is the set of all $g:\mathbb N\to\mathbb N$ such that
        there are some real $C>0$ and $C'>0$ and some $n_0\in\mathbb N$ such that
        for all $n\geq n_0$, $Cf(n)\leq g(n)\leq C'f(n)$.
    \end{itemize}
\end{definition}

Note that Definition \ref{bigodefn} does not measure growth rates, but rather
categorizes growth rates into Big-O and Big-$\Theta$ taxonomies.
For example, the same function can be
both $O(n^2)$ and $O(n^3)$, the former taxon being nested within the latter.

By Remark \ref{epiphanyremark}, Definition \ref{bigodefn} yields the following elegant
taxonomy of predictor intelligence.

\begin{definition}
\label{bigointelligencedefn}
    Suppose $p$ is a predictor, and suppose $f:\mathbb N\to\mathbb N$.
    \begin{itemize}
        \item
        We say \emph{$p$ has Big-O ASPI measure $O(f(n))$} if
        $p$ learns every evader $e$ such that $t_e$ is $O(f(n))$.
        \item
        We say \emph{$p$ has Big-$\Theta$ ASPI measure $\Theta(f(n))$} if
        $p$ learns every evader $e$ such that $t_e$ is $\Theta(f(n))$.
    \end{itemize}
\end{definition}

\section{ASPI measures and taxonomies using various number systems}
\label{exoticsection}

In this section we will consider various solutions to
Problem \ref{bigoproblem} in various different number systems,
and the ASPI measures and taxonomies they produce.

\subsection{The hyperreal ASPI taxonomy}
\label{hyperrealsubsection}

In this subsection, we will begin by considering growth rate \emph{comparison}, which is
a strictly simpler problem than growth rate \emph{measurement} (our proposed solution will
then lead to a numerical growth rate measure anyway). Given two functions $f$
and $g$, does $f$ outgrow $g$ or does $f$ not outgrow $g$? We would like to say that
$f$ outgrows $g$ if and only if $f(n)>g(n)$ for almost every $n\in\mathbb N$, but it is
not clear what ``almost every'' means. Certainly if $f(n)>g(n)$ for all but finitely many
$n\in\mathbb N$, it should be safe to say $f$ outgrows $g$, and if $f(n)\leq g(n)$ for
all but finitely many $n\in\mathbb N$, it should be safe to say $f$ does not outgrow
$g$. But what if there are infinitely many $n\in\mathbb N$ such that $f(n)>g(n)$, and
infinitely many $n\in\mathbb N$ such that $f(n)\leq g(n)$?

Adapting the key insight from \citet{alexander2019intelligence}, consider each $n\in\mathbb N$
to be a voter in an election to determine whether or not $f$ outgrows $g$.
Each $n$ votes based on whether or not $f(n)>g(n)$. For example, $532$ is a voter in this
election. If $f(532)>g(532)$, then $532$ casts her vote for ``$f$ outgrows $g$''; otherwise,
$532$ casts her vote for ``$f$ does not outgrow $g$''. This reduces the outgrowth problem
to an election decision problem: $f$ shall be considered to outgrow $g$ if and only if
``$f$ outgrows $g$'' gets a winning bloc of votes.

We need to decide what it means for a set $N\subseteq \mathbb N$ to constitute a winning
bloc of votes. We reason as follows.
\begin{itemize}
    \item $\emptyset$ should not be a winning bloc: if no-one votes for
    you, you lose.
    \item If $N_1$ is a winning bloc and $N_1\subseteq N_2$, then $N_2$ should
    be a winning bloc: if you were already winning, and additional voters switch
    their votes to you, you should still win.
    \item For any $N\subseteq \mathbb N$, either $N$ should be a winning bloc,
    or its complement $N^c=\mathbb N\backslash N$ should be a winning bloc:
    however the election goes, either you win or your opponent wins.
    \item We should insist on the outgrowth relation being transitive:
    if $f$ outgrows $g$ and $g$ outgrows $h$, then $f$ should outgrow $h$.
    Suppose that
    \begin{align*}
        N_{fg} &= \{ n\in\mathbb N \,:\, f(n)>g(n)\},\\
        N_{gh} &= \{ n\in\mathbb N \,:\, g(n)>h(n)\},\mbox{ and}\\
        N_{fh} &= \{ n\in\mathbb N \,:\, f(n)>h(n)\}.
    \end{align*}
    Clearly $N_{fg}\cap N_{gh}\subseteq N_{fh}$ but, a priori, we cannot say
    more: one can find functions $f,g,h$ such that
    $N_{fg}\cap N_{gh}=N_{fh}$. Thus, in order to ensure transitivity of the
    outgrowth relation, we should insist on the following requirement.
    Whenever $N_1$ and $N_2$ are winning blocs, then $N_1\cap N_2$ should be a winning bloc.
    \item We could trivially satisfy the
    above requirements, namely: we could choose some $n_0\in\mathbb N$ and
    declare that $N\subseteq \mathbb N$ is a winning bloc if and only if
    $n_0\in N$. In electoral terms, this would amount to making $n_0$ a dictator, whose
    vote decides the election regardless how anyone else votes. In terms of
    the outgrowth relation, this would amount to declaring that $f$ outgrows $g$
    if and only if $f(n_0)>g(n_0)$. That would be a poor method
    of comparing growth rates. Thus, we should insist that $\{n_0\}$ is not a winning
    bloc for any $n_0\in\mathbb N$.
\end{itemize}
Is it possible to satisfy all the above requirements, or are they too demanding?
It turns out it is possible. In fact, the above requirements are exactly the requirements
of a \emph{free ultrafilter}, an important device from mathematical logic.

\begin{definition}
\label{ultrafilterdefn}
    An \emph{ultrafilter on $\mathbb N$} (or more simply an \emph{ultrafilter})
    is a set $\mathcal U$ of subsets of $\mathbb N$ such that:
    \begin{enumerate}
        \item $\emptyset\not\in\mathcal U$.
        \item For every $N_1\in\mathcal U$, for every $N_2\subseteq \mathbb N$,
        if $N_1\subseteq N_2$, then $N_2\in\mathcal U$.
        \item For every $N\subseteq\mathbb N$, either $N\in\mathcal U$
        or $N^c\in\mathcal U$.
        \item ($\cap$-closure) For every $N_1,N_2\in\mathcal U$, $N_1\cap N_2\in \mathcal U$.
    \end{enumerate}
    An ultrafilter is \emph{free} if it does not contain any singleton
    $\{n_0\}$ ($n_0\in\mathbb N$).
\end{definition}

Clearly a free ultrafilter is exactly a notion of winning blocs meeting all our
requirements.
The following theorem is well-known in mathematical logic, and we state it without
proof.

\begin{theorem}
\label{freeultrafiltersexistthm}
    Free ultrafilters exist.
\end{theorem}

Theorem \ref{freeultrafiltersexistthm} is profound because
it is counter-intuitive that there should be a non-dictatorial method of determining
election winners satisfying $\cap$-closure. To see how counter-intuitive $\cap$-closure
is, suppose that in 2021 the Dog party wins the presidency and in 2022 the
Cat party wins the presidency (with the same voters every year and no other parties).
Call a voter a ``Dog-to-Cat switcher'' if they vote Dog in 2021 and Cat in 2022.
The $\cap$-closure property says in order to win in 2023, it would be enough to
get just the Dog-to-Cat switchers' votes \emph{and no others}.
For more on infinite-voter elections and free ultrafilters,
and especially their interplay with Arrow's impossibility theorem,
see \citet{kirman}.

For the remainder of the section, let $\mathcal U$ be a free ultrafilter.
Unfortunately, logicians have shown that, though free ultrafilters exist,
it is impossible to concretely exhibit one. More precisely, all known proofs
of Theorem \ref{freeultrafiltersexistthm} are non-constructive (using non-constructive
set-theoretic axioms such as the Axiom of Choice) and logicians have proven that
Theorem \ref{freeultrafiltersexistthm} cannot be proved constructively.

\begin{definition}
\label{outgrowsdefn}
    Suppose $f,g:\mathbb N\to\mathbb R$. We say $f$ \emph{outgrows} $g$ (according
    to $\mathcal U$) if
    \[\{n\in\mathbb N\,:\,f(n)>g(n)\}\in \mathcal U.\]
    In other words: if $\mathcal U$ is thought of as a black box deciding which
    subsets of $\mathbb N$ are winning blocs, then $f$ outgrows $g$ (according to
    $\mathcal U$) if and only if ``$f$ outgrows $g$'' wins the election when
    each $n\in\mathbb N$ votes for ``$f$ outgrows $g$'' or ``$f$ does not outgrow $g$''
    depending whether $f(n)>g(n)$ or $f(n)\leq g(n)$ respectively.
\end{definition}

We will now explain how Definition \ref{outgrowsdefn} leads to a numerical
growth rate measure and, in turn, an ASPI taxonomy.
We will show that by coming up with Definition \ref{outgrowsdefn}, we have actually
done most of the work of the construction of the so-called ultrapower construction of
the hyperreal number system, studied in the field of non-standard
analysis \citep{robinson, goldblatt2012lectures}.

\begin{definition}
\label{equivrelndefn}
(Compare Definition \ref{outgrowsdefn})
    Suppose $f,g:\mathbb N\to\mathbb R$. We say \emph{$f$ equals $g$ almost everywhere
    (according to $\mathcal U$)},
    written $f\equiv g$, if
    \[
    \{n\in\mathbb N\,:\, f(n) = g(n)\} \in \mathcal U.
    \]
    In other words, $f\equiv g$ if ``$f=g$'' wins the election (as decided by
    $\mathcal U$) when each
    $n\in\mathbb N$ votes for ``$f=g$'' or ``$f\not=g$'' depending whether
    $f(n)=g(n)$ or $f(n)\neq g(n)$ respectively.
\end{definition}

\begin{lemma}
    The relation $\equiv$ (from Definition \ref{equivrelndefn}) is an equivalence
    relation on the space $\mathbb R^{\mathbb N}$ of functions $\mathbb N\to\mathbb R$.
\end{lemma}

\begin{proof}
    Symmetry and reflexivity are trivial. To see that $\equiv$ is transitive, we will
    use the $\cap$-closure of $\mathcal U$. Suppose $f,g,h:\mathbb N\to\mathbb R$
    are such that $f\equiv g$ and $g\equiv h$, we must show $f\equiv h$.
    Let
    \begin{align*}
        N_{fg} &= \{n\in\mathbb N\,:\,f(n)=g(n)\},\\
        N_{gh} &= \{n\in\mathbb N\,:\,g(n)=h(n)\},\mbox{ and}\\
        N_{fh} &= \{n\in\mathbb N\,:\,f(n)=h(n)\}.\\
    \end{align*}
    Since $f\equiv g$, $N_{fg}\in\mathcal U$. Since $g\equiv h$, $N_{gh}\in\mathcal U$.
    By $\cap$-closure, $N_{fg}\cap N_{gh}\in\mathcal U$.
    Clearly $N_{fg}\cap N_{gh}\subseteq N_{fh}$, so, by (2) of
    Definition \ref{ultrafilterdefn},
    $N_{fh}\in \mathcal U$, that is, $f\equiv h$.
\end{proof}

\begin{definition}
\label{hyperrealsdefn}
    By the \emph{hyperreal numbers}, written $\mathbb R^*$,
    we mean the equivalence classes of $\equiv$.
    For every $f:\mathbb N\to\mathbb R$, write $\hat{f}$ for
    the hyperreal number (i.e., the $\equiv$-equivalence class) containing $f$.
    We endow $\mathbb R^*$ with arithmetic and order as follows
    (where $f,g:\mathbb N\to\mathbb R$):
    \begin{itemize}
        \item We define addition on $\mathbb R^*$ by declaring that
        $\hat{f}+\hat{g}=\hat{h}$
        where $h(n)=f(n)+g(n)$.
        \item We define multiplication on $\mathbb R^*$ by declaring that
        $\hat{f}\cdot \hat{g}=\hat{h}$
        where $h(n)=f(n)g(n)$.
        \item We order $\mathbb R^*$ by declaring that
        $\hat{f}>\hat{g}$ if and only if $f$ outgrows $g$ (Definition \ref{outgrowsdefn}).
    \end{itemize}
\end{definition}

The following is well-known and we state it
without proof.

\begin{theorem}
    The addition, multiplication, and ordering in Definition \ref{hyperrealsdefn}
    are well-defined, and they make $\mathbb R^*$ an ordered field.
\end{theorem}

Having invested in this machinery,
we now have a trivial hyperreal solution to Problem \ref{bigoproblem}.

\begin{definition}
\label{hyperrealgrowthratedefn}
    (The hyperreal solution to Problem \ref{bigoproblem})
    For any function $f:\mathbb N\to\mathbb N$, the \emph{hyperreal growth rate
    of $f$} is the hyperreal number $\hat f$.
\end{definition}

Because of the non-constructive nature of free ultrafilters, the following notion
is computationally impractical. However, it could potentially be useful for
proving theoretical properties about the intelligence of predictors. In the following
definition, rather than assigning a particular hyperreal number intelligence to every
predictor, rather, we categorize predictors into a taxonomy.
This is necessary because there is no way of choosing canonical bounds
of bounded sets of hyperreal numbers in general. For lack of a way of
choosing a particular bound, we are forced to consider many taxa corresponding
to many bounds.

\begin{definition}
\label{hyperrealhibbardintelligencedefn}
    (The hyperreal ASPI taxonomy)
    Let $p$ be a predictor and let $\hat f$ be a hyperreal number.
    We say that $p$ \emph{has hyperreal ASPI intelligence at least $\hat f$}
    if and only if the following condition holds:
    for every evader $e$, if the hyperreal growth rate of $t_e$ is
    $<\hat f$, then $p$ learns $e$.
\end{definition}

\subsection{The surreal ASPI measure}

The surreal numbers \citep{conway, knuth, ehrlich2012absolute}
are an even larger extension
of the real numbers
into which the hyperreals can be embedded.

\begin{definition}
    (The surreal solution to Problem \ref{bigoproblem})
    Let $\iota$ be the embedding of the hyperreal numbers into the surreal
    numbers.
    For any function $f:\mathbb N\to\mathbb N$, the \emph{surreal growth rate
    of $f$} is $\iota(\hat r)$ where $\hat r$ is the hyperreal growth
    rate of $f$ (Definition \ref{hyperrealgrowthratedefn}).
\end{definition}

One advantage of the surreal numbers is that for any set $L$ of surreal numbers,
there is a canonical way to choose a surreal strict upper bound on $L$,
which upper bound is written $\{L\,|\,\}$. This upper bound is similar to a
supremum of $L$ in a certain sense: the surreal numbers possess a
simplicity-hierarchical structure, and $\{L\,|\,\}$ is the \emph{simplest}
strict upper bound of $L$. This allows us to
exactly measure intelligence of predictors,
rather than merely classify predictors in a taxonomy as in Definition \ref{hyperrealhibbardintelligencedefn}.

\begin{definition}
\label{surrealhibbardintelligencedefn}
    (The surreal ASPI measure)
    For every predictor $p$, the \emph{surreal ASPI measure} of $p$
    is defined to be $\{L\,|\,\}$, the simplest surreal strict upper bound
    of $L$, where $L$ is the set of all surreal numbers $\ell$ such that
    the following condition holds: for every evader $e$,
    if the surreal growth rate of $t_e$ is $<\ell$, then $p$ learns $e$.
\end{definition}

\subsection{ASPI measures based on majorization hierarchies}

Majorization hierarchies \citep{weiermann2002slow}
provide ordinal-number-valued measures for the growth
rates of certain functions. A majorization hierarchy depends
on many infinite-dimensional parameters. We will describe two
majorization hierarchies up to the ordinal $\epsilon_0$,
using standard choices for the parameters, and the ASPI measures
which they produce.

\begin{definition}
    (Classification of ordinal numbers)
    Ordinal numbers are divided into three types:
    \begin{enumerate}
        \item Zero: The ordinal $0$.
        \item Successor ordinals: Ordinals of the form $\alpha+1$ for some ordinal $\alpha$.
        \item Limit ordinals: Ordinals which are not successor ordinals nor $0$.
    \end{enumerate}
\end{definition}

For example, the smallest infinite ordinal, $\omega$, is a limit ordinal. It is not zero
(because zero is finite),
nor can it be a successor ordinal, because if it were a successor ordinal, say, $\alpha+1$,
then $\alpha$ would be finite (since $\omega$ is the \emph{smallest} infinite ordinal),
but then $\alpha+1$ would be finite as well.

Ordinal numbers have an arithmetical structure: two ordinals $\alpha$ and $\beta$
have a sum $\alpha+\beta$, a product $\alpha\cdot \beta$, and a power
$\alpha^\beta$. It would be beyond the scope of this paper to give the full
definition of these operations. We will only remark that some care is needed because
although ordinal arithmetic is associative---e.g.,
$(\alpha+\beta)+\gamma=\alpha+(\beta+\gamma)$, and similarly for multiplication---it is
not generally commutative: $\alpha+\beta$ is not always equal to $\beta+\alpha$,
and $\alpha\cdot\beta$ is not always equal to $\beta\cdot\alpha$. For this reason,
one often sees products like $\alpha\cdot 2$, which are not necessarily equivalent to the
more familiar $2\cdot\alpha$.

The ordinal $\epsilon_0$ is the smallest ordinal bigger than the ordinals
$\omega,\omega^\omega,\omega^{\omega^\omega},\ldots$. It satisfies the equation
$\epsilon_0=\omega^{\epsilon_0}$ and can be intuitively thought of as
\[
    \epsilon_0 = \omega^{\omega^{\omega^{\iddots}}}.
\]
Ordinals below $\epsilon_0$ include such ordinals as $\omega$,
$\omega^{\omega+1}+\omega^{\omega}+\omega^5+3$,
\[
\omega^{\omega^{\omega^{\omega^\omega}}}+
\omega^{\omega^{\omega^\omega}+\omega^{\omega\cdot 2+1}+\omega^4 + 3}
+ \omega^{\omega^5+\omega^3}+\omega^8+1,
\]
and so on.
Any ordinal below $\epsilon_0$ can be uniquely written in the form
\[
    \omega^{\alpha_1}+\omega^{\alpha_2}+\cdots + \omega^{\alpha_k}
\]
where $\alpha_1\geq\cdots\geq\alpha_k$ are smaller ordinals below $\epsilon_0$---this form
for an ordinal below $\epsilon_0$ is called its \emph{Cantor normal form}.
For example, the Cantor normal form for $\omega^{\omega\cdot 2}\cdot 2+\omega\cdot 3+2$
is
\[
\omega^{\omega\cdot 2}\cdot 2+\omega\cdot 3+2
=
\omega^{\omega\cdot 2} + \omega^{\omega\cdot 2} + \omega^1 + \omega^1 + \omega^1
+\omega^0 + \omega^0.
\]

\begin{definition}
\label{fundsequencesdefn}
    (Standard fundamental sequences for limit ordinals $\leq\epsilon_0$)
    Suppose $\lambda$ is a limit ordinal $\leq\epsilon_0$. We define a
    \emph{fundamental sequence for $\lambda$},
    written $(\lambda[0],\lambda[1],\lambda[2],\ldots)$, inductively as follows.
    \begin{itemize}
        \item
        If $\lambda=\epsilon_0$, then $\lambda[0]=0$,
        $\lambda[1]=\omega^0$, $\lambda[2]=\omega^{\omega^0}$,
        $\lambda[3]=\omega^{\omega^{\omega^0}}$,
        and so on.
        \item
        If $\lambda$ has Cantor normal form
        $\omega^{\alpha_1}+\cdots+\omega^{\alpha_k}$ where $k>1$,
        then
        each
        \[
            \lambda[i] = \omega^{\alpha_1}+\cdots+\omega^{\alpha_{k-1}}
            + (\omega^{\alpha_k}[i]).
        \]
        \item
        If $\lambda$ has Cantor normal form $\omega^{\alpha+1}$,
        then each $\lambda[i]=\omega^{\alpha}\cdot i$.
        \item
        If $\lambda$ has Cantor normal form $\omega^{\lambda_0}$ where $\lambda_0$
        is a limit ordinal, then each $\lambda[i]=\omega^{\lambda_0[i]}$.
    \end{itemize}
\end{definition}

\begin{example}
    (Fundamental sequence examples)
    \begin{itemize}
        \item
        The fundamental sequence for $\lambda=\omega=\omega^1=\omega^{0+1}$ is
        $\omega^0\cdot 0, \omega^0\cdot 1, \omega^0\cdot 2, \ldots$,
        i.e., $0, 1, 2, \ldots$.
        \item
        The fundamental sequence for $\lambda=\omega^5$ is
        $0,\omega^4,\omega^4\cdot 2,\omega^4\cdot 3,\ldots$.
        \item
        The fundamental sequence for $\lambda=\omega^\omega$ is
        $\omega^0,\omega^1,\omega^2,\ldots$.
        \item
        The fundamental sequence for $\lambda=\omega^\omega+\omega$ is
        $\omega^\omega+0,\omega^\omega+1,\omega^\omega+2,\ldots$.
    \end{itemize}
\end{example}

\begin{definition}
\label{slowgrowinghierarchydefn}
    (The standard slow-growing hierarchy up to $\epsilon_0$)
    We define functions $g_\beta:\mathbb N\to\mathbb N$ (for all ordinals
    $\beta\leq \epsilon_0$) by transfinite induction as follows.
    \begin{itemize}
        \item
        $g_0(n)=0$.
        \item
        $g_{\alpha+1}(n) = g_\alpha(n) + 1$ if $\alpha+1\leq\epsilon_0$.
        \item
        $g_{\lambda}(n) = g_{\lambda[n]}(n)$ if $\lambda\leq\epsilon_0$ is a limit ordinal.
    \end{itemize}
\end{definition}

Here are some early levels in the slow-growing hierarchy, spelled out in detail.

\begin{example}
\label{highdetailslowgrowingexample}
    (Early examples of functions in the slow-growing hierarchy)
    \begin{enumerate}
        \item
        $g_1(n)=g_{0+1}(n)=g_0(n)+1=0+1=1$.
        \item
        $g_2(n)=g_{1+1}(n)=g_1(n)+1=1+1=2$.
        \item
        More generally, for all $m\in\mathbb N$,
        $g_m(n)=m$.
        \item
        $g_\omega(n)=g_{\omega[n]}(n)=g_n(n)=n$.
        \item
        $g_{\omega+1}(n)=g_{\omega}(n)+1=n+1$.
        \item
        $g_{\omega+2}(n)=g_{\omega+1}(n)+1=(n+1)+1=n+2$.
        \item
        More generally, for all $m\in\mathbb N$,
        $g_{\omega+m}(n)=n+m$.
        \item
        $g_{\omega\cdot 2}(n)=g_{(\omega\cdot 2)[n]}(n)
        =g_{\omega+n}(n)=n+n=n\cdot 2$.
    \end{enumerate}
\end{example}

Following Example \ref{highdetailslowgrowingexample}, the reader should be able
to fill in the details in the following example.

\begin{example}
    (More examples from the slow-growing hierarchy)
    \begin{enumerate}
        \item
        $g_{\omega^2}(n)=n^2$.
        \item
        $g_{\omega^3}(n)=n^3$.
        \item
        $g_{\omega^\omega}(n)=n^n$.
        \item
        $g_{\omega^{\omega\cdot 3+1}+\omega+5}(n)=n^{3n+1}+n+5$.
        \item
        $g_{\omega^{\omega^{\omega}}}(n)=n^{n^n}$.
    \end{enumerate}
\end{example}

What about $g_{\epsilon_0}$? Thinking of $\epsilon_0$ as
\[\omega^{\omega^{\omega^{\iddots}}},\]
one might expect $g_{\epsilon_0}(n)$ to be
\[n^{n^{n^{\iddots}}},\]
but such an infinite tower
of natural number exponents makes no sense if $n>1$. Instead,
the answer defies familiar mathematical
notation.

\begin{example}
\label{epsilon0example}
(Level $\epsilon_0$ in the slow-growing hierarchy)
The values of $g_{\epsilon_0}$ are as follows:
\begin{itemize}
    \item
    $g_{\epsilon_0}(0)=0$.
    \item
    $g_{\epsilon_0}(1)=1^1$.
    \item
    $g_{\epsilon_0}(2)=2^{2^2}$.
    \item
    $g_{\epsilon_0}(3)=3^{3^{3^3}}$.
    \item
    And so on.
\end{itemize}
\end{example}

Examples \ref{highdetailslowgrowingexample}--\ref{epsilon0example} illustrate
how the slow-growing hierarchy systematically provides a family of reference
functions against which any particular function can be compared.
This yields a solution to Problem \ref{bigoproblem}: we can declare the growth
rate of an arbitrary function $f:\mathbb N\to\mathbb N$ to be the smallest ordinal
$\beta< \epsilon_0$ such that $g_\beta\succ f$ (or $\infty$ if there is no such
$\beta$).
For any bounded set $S$ of ordinals, there is a canonical upper bound for $S$,
namely, the supremum of $S$.
Thus we obtain an ASPI measure (not just a taxonomy).

\begin{definition}
\label{tradmajorizationhierarchyhibbardmeasuredefn}
    If $p$ is a predictor, the \emph{ASPI measure of $p$ given by the
    standard slow-growing hierarchy up to $\epsilon_0$} is defined to be the
    supremum of $S$ (or $\infty$
    if $\epsilon_0\in S$), where $S$ is the set of all ordinals
    $\alpha\leq\epsilon_0$
    such that the following condition holds:
    for every predictor $e$, if $g_\alpha\succ t_e$, then $p$ learns $e$.
\end{definition}

In Definition \ref{slowgrowinghierarchydefn}, in the successor ordinal case,
we chose to define $g_{\alpha+1}(n)=g_\alpha(n)+1$. The resulting majorization
hierarchy is referred to as \emph{slow-growing} because in some sense this
makes $g_{\alpha+1}$ just barely faster-growing than $g_\alpha$.
Different definitions of $g_{\alpha+1}$ would yield different majorization
hierarchies, such as the following.

\begin{definition}
\label{fastgrowinghierarchydefn}
    (The standard fast-growing hierarchy up to $\epsilon_0$, also known as
    the Wainer hierarchy)
    We define functions $h_\beta:\mathbb N\to\mathbb N$ (for all ordinals
    $\beta\leq \epsilon_0$) by transfinite induction as follows.
    \begin{itemize}
        \item
        $h_0(n)=0$.
        \item
        $h_{\alpha+1}(n) = h^n_\alpha(n)$, where $h^n_\alpha$ is the $n$th
        iterate of $h_\alpha$ (so $h^1_\alpha(x)=h_\alpha(x)$,
        $h^2_\alpha(x)=h_\alpha(h_\alpha(x))$,
        $h^3_\alpha(x)=h_\alpha(h_\alpha(h_\alpha(x)))$, and so on).
        \item
        $h_{\lambda}(n) = h_{\lambda[n]}(n)$ if $\lambda$ is a
        limit ordinal $\leq\epsilon_0$.
    \end{itemize}
\end{definition}

The functions in the fast-growing hierarchy grow quickly
as $\alpha$ grows. It can be shown \citep{wainer1987provably} that
for every computable function $f$ whose totality can be proven from the axioms of
Peano arithmetic, there is some $\alpha<\epsilon_0$ such that $h_\alpha\succ f$.

\begin{definition}
\label{fastmajorizationhierarchyhibbardmeasuredefn}
    If $p$ is a predictor, the \emph{ASPI measure of $p$ given by the
    standard fast-growing hierarchy up to $\epsilon_0$} is defined to be the
    supremum of $S$ (or $\infty$ if $\epsilon_0\in S$),
    where $S$ is the set of all ordinals $\alpha\leq\epsilon_0$ such that
    the following condition holds:
    for every predictor $e$, if $h_\alpha\succ t_e$, then $p$ learns $e$.
\end{definition}

Between Definitions \ref{tradmajorizationhierarchyhibbardmeasuredefn} and
\ref{fastmajorizationhierarchyhibbardmeasuredefn}, the former offers a finer
granularity
intelligence measure for the predictors to which it assigns non-$\infty$
intelligence, but the latter assigns non-$\infty$ intelligence to a larger
set of predictors.

Definitions \ref{slowgrowinghierarchydefn}
and \ref{fastgrowinghierarchydefn} are only two
examples of
majorization hierarchies. Both the slow- and fast-growing hierarchies can be
extended by extending the fundamental sequences of Definition
\ref{fundsequencesdefn} to larger ordinals\footnote{Remarkably,
the slow-growing hierarchy eventually catches up with the fast-growing hierarchy
if both hierarchies are extended to sufficiently large ordinals
\citep{wainer1989slow, girard1981pi12}, a beautiful illustration of
how counter-intuitive
large ordinal numbers can be.}, however, the larger
the ordinals become, the more difficult it is to do this, and especially the less
clear it is how to do it in any sort of canonical way.
There are also other choices for how to proceed at successor ordinal stages besides
$g_{\alpha+1}(n)=g_\alpha(n)+1$ or $h_{\alpha+1}(n)=h^n_\alpha(n)$---for example,
one of the oldest majorization hierarchies is the Hardy hierarchy
\citep{hardy1904theorem}, where $H_{\alpha+1}(n)=H_\alpha(n+1)$.
And even for ordinals up to $\epsilon_0$,
there are other ways to choose fundamental sequences besides how we defined them in
Definition \ref{fundsequencesdefn}---choosing non-canonical fundamental sequences can
drastically alter the resulting majorization hierarchy \citep{weiermann1997sometimes}.
All these different majorization hierarchies yield different ASPI measures.

\subsection{A remark about ASPI measures and AGI intelligence}

All the ASPI measures and taxonomies we have defined so far double as indirect
intelligence measures and taxonomies for an AGI, by the argument we made
in Subsection \ref{agiproxysection}.

For a given AGI $X$, a priori, we cannot say much about the predictor which $X$
would act as if $X$ were commanded to act as a predictor. But there is one particularly
elegant and parsimonious strategy which $X$ might use, a \emph{brute force strategy},
namely:
\begin{itemize}
    \item Enumerate all the
    computable functions $f$ which $X$ knows to be total, and for each one, attempt to
    predict the evader $e$ by assuming that the evader's runtime $t_e$
    satisfies $f\succ t_e$.
    If the evader proves not to be so majorized (by differing from every computable
    function whose runtime is so majorized), then move on to the next
    known total function $f$, and continue the process.
\end{itemize}
We do not
know for certain which predictor $X$ would imitate when commanded to act as a predictor,
but it seems plausible that $X$ would use this brute force strategy or something
equivalent.

For an AGI $X$ who uses the above brute force strategy, ASPI
measures of $X$'s intelligence would be determined by $X$'s knowledge, namely,
by the runtime complexity of the computable functions which $X$ knows to be total.
Furthermore, the most natural way for $X$ to know totality of functions with large
runtime complexity, is for $X$ to know fundamental sequences for large ordinal
numbers, and produce said functions by means of majorization
hierarchies\footnote{It may be
possible for an AGI to be contrived to know totality of functions that are larger
than the functions produced by majorization hierarchies up to ordinals the AGI knows
about, but we conjecture that that is not the case for AGIs not so deliberately
contrived.}. This suggests a connection between
\begin{enumerate}
    \item
    ASPI measures like that of
    Definition \ref{fastmajorizationhierarchyhibbardmeasuredefn}, and
    \item
    intelligence measures based on which ordinals the AGI knows
    \citep{ioi1}.
\end{enumerate}
Indeed, \citet{ioi2} has argued that
the task of notating large ordinals is one which
spans the entire range of intelligence.
This is reminiscent of Chaitin's proposal to use ordinal notation
as a goal intended to facilitate evolution---``and the larger the ordinal,
the fitter the organism'' \citep{chaitin}---and Good's observation
\citep{good1969godel} that iterated Lucas-Penrose contests boil down to
contests to name the larger ordinal.


\section{Pros and cons of different ASPI measures and taxonomies}
\label{prosandconssection}

Here are pros and cons of the ASPI measures and taxonomies which arise
from different solutions
to the problem (Problem \ref{bigoproblem}) of measuring the growth rate of functions.

\begin{itemize}
    \item
    The original Hibbard measure (Definition \ref{classichibbardmeasuredefn}),
    which arises by measuring growth rate by comparing
    a function with Liu's enumeration \citep{liu1960enumeration} of the primitive
    recursive functions:
    \begin{itemize}
        \item
        Pro: Relatively concrete.
        \item
        Pro: Measures intelligence using a familiar number system (the natural numbers).
        \item
        Con: The numbers which the measure outputs are not very meaningful, in
        that predictor $p$ having a measure of
        $+1$ higher than predictor $q$ tells us little
        about how \emph{much} more computationally complex the evaders which $p$
        learns are, versus the evaders which $q$ learns.
        \item
        Con: Only distinguishes sufficiently non-intelligent predictors; all predictors
        sufficiently intelligent receive measure $\infty$.
    \end{itemize}
    \item
    Big-O/Big-$\Theta$ (Definition \ref{bigointelligencedefn}),
    in which, rather than directly measuring the intelligence of a predictor, instead, we
    would talk of a predictor's intelligence being $O(f(n))$ or $\Theta(f(n))$
    for various functions $f:\mathbb N\to\mathbb N$:
    \begin{itemize}
        \item
        Pro: Nearly perfect granularity (slightly coarser than perfect granularity because
        of the constants $C,C'$ in Definition \ref{bigodefn}).
        \item
        Pro: Computer scientists already use Big-O/Big-$\Theta$ routinely
        and are comfortable with them.
        \item
        Con: A non-numerical taxonomy.
    \end{itemize}
    \item
    Hyperreal intelligence (Definition \ref{hyperrealhibbardintelligencedefn}):
    \begin{itemize}
        \item
        Pro: A taxonomy like Big-O/Big-$\Theta$, but with the added benefit
        that the taxons are numerical.
        \item
        Pro: Perfect granularity.
        \item
        Con: Depends on a free ultrafilter (rendering it
        computationally impractical).
    \end{itemize}
    \item
    Surreal intelligence (Definition \ref{surrealhibbardintelligencedefn}):
    \begin{itemize}
        \item
        Pro: An actual numerical measure (not just a taxonomy), with
        perfect granularity.
        \item
        Con: The numbers which the measure outputs are surreal numbers,
        which are relatively new and thus unfamiliar, and are difficult
        to work with in practice.
    \end{itemize}
    \item
    Intelligence based on a majorization hierarchy such as the
    standard slow- or fast-growing hierarchy up to $\epsilon_0$
    (Definitions \ref{tradmajorizationhierarchyhibbardmeasuredefn}
    and \ref{fastmajorizationhierarchyhibbardmeasuredefn}):
    \begin{itemize}
        \item
        Pro: A numerical measure, albeit less granular than the
        Big-O/Big-$\Theta$ taxonomies.
        \item
        Pro: Relatively concrete.
        \item
        Pro: The numbers which the measure outputs are meaningful, in the sense that
        the degree to which a predictor $p$ is more intelligent than a
        predictor $q$ is reflected
        in the degree to which $p$'s intelligence-measure is larger than $q$'s.
        \item
        Con: The numbers which the measure outputs are ordinal numbers, which may be
        unfamiliar to some users.
        \item
        Con: Only distinguishes sufficiently non-intelligent predictors; for any particular
        majorization hierarchy, all predictors
        sufficiently intelligent receive measure $\infty$.
    \end{itemize}
\end{itemize}

\section{Conclusion}
\label{conclusionsection}

To summarize:
\begin{itemize}
    \item
    \citet{hibbard} proposed an intelligence measure for predictors
    in games of adversarial sequence prediction.
    \item
    We argued that Hibbard's idea actually splits into two orthogonal sub-ideas.
    First: that intelligence can be measured via the growth-rates of the run-times
    of evaders that a predictor can learn to predict. Second: that such growth-rates can
    be measured in one specific way (involving an enumeration of the primitive recursive
    functions). We argued that there are many other ways to measure growth-rates,
    and that each method of measuring growth-rates yields a corresponding
    adversarial sequence prediction intelligence (ASPI) measure or taxonomy.
    \item
    We considered several specific ways of measuring growth-rates of functions, and exhibited
    corresponding ASPI measures and taxonomies. The growth-rate-measuring methods
    which we considered were: Big-O/Big-$\Theta$ notation; hyperreal numbers;
    surreal numbers; and majorization hierarchies.
    \item
    We also discussed how the intelligence of adversarial sequence predictors
    can be considered as an approximation of the intelligence of idealized AGIs.
\end{itemize}

\section*{Acknowledgments}

We acknowledge Bryan Dawson for feedback on Section \ref{hyperrealsubsection}.
We acknowledge Philip Ehrlich for correcting a mistake.
We acknowledge Roman Yampolskiy for providing literature references.
We acknowledge the editor and the reviewers for much generous feedback and suggestions.

%\bibliographystyle{plain}
\bibliography{hibbard}
\end{document}
