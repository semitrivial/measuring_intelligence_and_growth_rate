\documentclass{article}
\usepackage[utf8]{inputenc}
\usepackage{natbib}
\usepackage{amssymb}
\usepackage{amsthm}
\usepackage{amsmath}
\usepackage{mathrsfs}
\usepackage{mathdots}
\newtheorem{theorem}{Theorem}
\newtheorem{definition}[theorem]{Definition}
\newtheorem{corollary}[theorem]{Corollary}
\newtheorem{example}[theorem]{Example}
\newtheorem{assumption}[theorem]{Assumption}
\newtheorem{lemma}[theorem]{Lemma}
\newtheorem{problem}[theorem]{Problem}
\newtheorem{proposition}[theorem]{Proposition}
\newtheorem{informallemma}[theorem]{Informal Lemma}
\newtheorem{remark}[theorem]{Remark}
\newtheorem{openquestion}[theorem]{Open Question}

\title{Measuring intelligence and growth rate: variations on
Hibbard's intelligence measure}
\author{Samuel Alexander\thanks{The U.S.\ Securities and Exchange Commission}}
\date{2020}

\begin{document}

\maketitle

\begin{abstract}
    In 2011, Hibbard suggested an intelligence measure for agents
    who compete in an adversarial sequence prediction game. We argue
    that Hibbard's idea should actually be considered as two separate
    ideas: first, that the intelligence of such agents can be measured
    based on the growth rates of the runtimes of the competitors that
    they defeat; and second, one specific (somewhat arbitrary) method for measuring said
    growth rates. Whereas Hibbard's intelligence measure is based on the latter
    growth-rate-measuring method, we survey
    standard methods for measuring function
    growth rates, and exhibit the Hibbard-like intelligence measures
    which result from those. Of particular interest, we obtain intelligence measures
    based on Big-O notation and similar notation systems, which measures
    are novel in that they challenge conventional notions of what an
    intelligence measure should look like.
\end{abstract}

\section{Introduction}

In his insightful paper \cite{hibbard}, Bill Hibbard introduces a novel
intelligence measure (which we will here refer to as the \emph{original Hibbard measure})
for agents who play a game of adversarial sequence prediction
\cite{hibbard2008adversarial}
``against a hierarchy of increasingly difficult sets of'' evaders (environments that attempt
to emit $1$s and $0$s in such a way as to evade prediction).
The levels of Hibbard's hierarchy are labelled by natural numbers, and
an agent's original Hibbard measure is the maximum $n\in\mathbb N$ such that
said agent learns to predict all the evaders in the $n$th level of the hierarchy,
or implicitly\footnote{Hibbard does not explicitly include the $\infty$ case in his
definition, but in his Proposition 3 he refers to agents having ``finite intelligence'', and
it is clear from context that by this he means agents who fail to predict some evader
somewhere in the hierarchy.} an agent's original Hibbard measure is $\infty$
if said agent learns to predict all the evaders in all levels of Hibbard's hierarchy.

The hierarchy which Hibbard uses to measure intelligence is based on the growth
rate of the runtime complexity of evaders.
We will argue that Hibbard's idea is really a combination of two
orthogonal ideas. First: that in some sense the intelligence of a predicting agent
can be measured based on the growth rates of the runtimes of the evaders whom that
predictor learns to predict. Second: Hibbard proposed one particular method for
measuring said growth rates. The growth rate measurement which Hibbard proposed yields
a corresponding intelligence measure for these agents. We will argue that \emph{any}
method for measuring growth rates of functions yields a corresponding Hibbard measure.

The particular method which Hibbard used to measure function growth rates is
not very standard, nor (in our opinion) very useful. We will survey standard
ways of measuring function growth rates, and these will yield corresponding
Hibbard-like intelligence measures.

The structure of the paper is as follows.
\begin{itemize}
    \item
    In Section \ref{originalmeasuresection}, we review the original Hibbard measure.
    \item
    In Section \ref{growthratesection}, we argue that any method of measuring
    the growth rate of functions yields a Hibbard-like intelligence measure,
    and that the original Hibbard measure is just a special case resulting from
    one particular method of measuring function growth rate.
    \item
    In Section \ref{bigosection}, we consider the most standard methods of measuring
    function growth rate---Big-O notation, Big-$\Theta$ notation, and Big-$\Omega$
    notation---and define corresponding predictor intelligence taxonomies.
    \item
    In Section \ref{exoticsection}, we consider several numeric solutions to
    the problem of measuring the growth rate of functions (which use various extended number
    systems), and define
    corresponding Hibbard-like measures of predictor intelligence.
    \item
    In Section \ref{prosandconssection}, we give pros and cons of different
    Hibbard-style measures.
    \item
    In Section \ref{conclusionsection}, we summarize and make concluding remarks.
\end{itemize}

\section{Hibbard's original measure}
\label{originalmeasuresection}

Hibbard proposed an intelligence measure for measuring the intelligence
of agents who compete to predict evaders in a game of
adversarial sequence prediction (we define this
formally below). A predictor $p$ (whose intelligence we want to measure)
competes against evaders $e$. In each step of the game,
both predictor and evader simultaneoulsy choose a binary digit, $1$ or $0$.
Only after both of them have made their choice do they see which choice the other
one made, and then the game proceeds to the next step. The predictor's goal in
each step of the game is to choose the same digit that the evader will choose;
the evader's goal is to choose a different digit than the predictor. The predictor
wins the game (and is said to \emph{learn to predict $e$}, or simply to
\emph{learn $e$}) if, after finitely many
initial steps, eventually the predictor always chooses the same digit as the
evader.

\begin{definition}
By $B$, we mean the binary alphabet $\{0,1\}$. By $B^*$, we mean the set of all
finite binary sequences. By $\langle\rangle$ we mean the empty binary sequence.
\end{definition}

\begin{definition}
\label{evaderpredictordefn}
    (Predictors and evaders)
    \begin{enumerate}
        \item
        By a \emph{predictor}, we mean a Turing machine $p$
        which takes as input a finite (possibly empty) binary sequence
        $(x_1,\ldots,x_n)\in B^*$
        (thought of as a sequence of \emph{evasions})
        and outputs $0$ or $1$ (thought of as a \emph{prediction}), which output
        we write as $p(x_1,\ldots,x_n)$.
        \item
        By an \emph{evader}, we mean a Turing machine $e$
        which takes as input a finite (possibly empty) binary sequence
        $(y_1,\ldots,y_n)\in B^*$
        (thought of as a sequence of \emph{predictions})
        and outputs $0$ or $1$ (thought of as an \emph{evasion}), which output
        we write as $e(y_1,\ldots,y_n)$.
        \item
        For any predictor $p$ and evader $e$, the \emph{result of $p$ playing the
        game of adversarial prediction against $e$} (or more simply, the \emph{result of
        $p$ playing against $e$}) is the infinite binary sequence
        $(x_1,y_1,x_2,y_2,\ldots)$
        defined as follows:
        \begin{enumerate}
            \item
            The first evasion
            $x_1=e(\langle\rangle)$ is
            the output of $e$ when run on the empty prediction-sequence.
            \item
            The first prediction
            $y_1=p(\langle\rangle)$ is
            the result of applying $p$ to the empty evasion-sequence.
            \item
            For all $n>0$, the $(n+1)$th evasion
            $x_{n+1}=e(y_1,\ldots,y_n)$ is
            the output of $e$ on the sequence of the first $n$ predictions.
            \item
            For all $n>0$, the $(n+1)$th prediction
            $y_{n+1}=p(x_1,\ldots,x_n)$ is
            the result of applying $p$ to the first $n$ evasions.
        \end{enumerate}
        \item
        Suppose $r=(x_1,y_1,x_2,y_2,\ldots)$ is the result of a predictor $p$ playing
        against an evader $e$. For every $n\geq 1$,
        we say \emph{the predictor wins round $n$ in $r$}
        if $x_n=y_n$; otherwise,
        \emph{the evader wins round $n$ in $r$}.
        We say that \emph{$p$ learns to predict $e$}
        (or simply that \emph{$p$ learns $e$}) if there is some $N\in\mathbb N$
        such that for all $n>N$, $p$ is the winner of round $n$ in $r$.
    \end{enumerate}
\end{definition}

Note that if $e$ simply ignores its inputs $(y_1,\ldots,y_n)$ and instead
computes $e(y_1,\ldots,y_n)$ based only on $n$, then $e$ is essentially a sequence.
Thus Definition \ref{evaderpredictordefn} is a generalization of sequence prediction,
which many authors have written about (such as Legg \cite{legg2006there}, who gives many
references).

In the following definition, we differ from Hibbard's original paper
because of a minor (and fortunately, easy-to-fix) error there.

\begin{definition}
\label{tsubedefinition}
    Suppose $e$ is an evader.
    For each $n\in\mathbb N$, let $t_e(n)$ be the maximum number of steps that $e$ takes
    to run on any length-$n$ sequence of binary digits.
    In other words, $t_e(0)$ is the number of steps $e$ takes to run on $\langle\rangle$,
    and for all $n>0$,
    \[
        t_e(n) = \max_{b_1,\ldots,b_n\in \{0,1\}}
        (\text{number of steps $e$ takes to run on $(b_1,\ldots,b_n)$}).
    \]
\end{definition}

\begin{example}
    Let $e$ be an evader. Then
    $t_e(2)$ is equal to the number of steps $e$ takes to run on input
    $(0,0)$, or to run on input $(0,1)$, or to run on input $(1,0)$, or to run on input
    $(1,1)$---whichever of these four possibilities is largest.
\end{example}

\begin{definition}
\label{functionsuccdefn}
    Suppose $f:\mathbb N\to\mathbb N$ and $g:\mathbb N\to\mathbb N$.
    We say $f\succ g$ if there is some $n_0\in \mathbb N$ such that for all
    $n>n_0$, $f(n)>g(n)$.
\end{definition}

\begin{definition}
\label{evadersetdefinition}
    Suppose $f:\mathbb N\to\mathbb N$. We define
    $E_f$ to be the set of all evaders $e$ such that
    $f\succ t_e$.
\end{definition}

\begin{definition}
\label{classichibbardmeasuredefn}
    (The original Hibbard measure)
    Let $g_1,g_2,\ldots$ be the enumeration of the primitive recursive
    functions given by Liu \cite{liu1960enumeration}.
    For each $m>0$, define $f_m:\mathbb N\to\mathbb N$ by
    \[f_m(k)=\max_{0<i\leq m}\max_{j\leq k}g_i(j).\]
    For any predictor $p$, we define the \emph{original Hibbard intelligence of $p$}
    to be the maximum $m>0$
    such that $p$ learns to predict $e$ for every $e\in E_{f_m}$
    (or $0$ if there is no such $m$, or $\infty$ if $p$ learns to predict $e$
    for every $e\in E_{f_m}$ for every $m>0$).
\end{definition}


\section{Quantifying growth rates of functions}
\label{growthratesection}

The following is a very general and open-ended problem.

\begin{problem}
\label{bigoproblem}
    Quantify the growth-rate of functions from $\mathbb N$ to $\mathbb N$.
\end{problem}

The definition of the original Hibbard measure (Definition \ref{classichibbardmeasuredefn})
can be thought of as implicitly depending on a specific solution to Problem
\ref{bigoproblem}, which we make explicit in the following definition.

\begin{definition}
\label{hibbardgrowthratedefn}
    For each $m>0$, let $f_m$ be as in Definition \ref{classichibbardmeasuredefn}.
    For each $f:\mathbb N\to\mathbb N$, we define the \emph{original Hibbard growth rate}
    $H(f)$ to be $\min\{m>0\,:\,f_m\succ f\}$ if there is any such $m>0$, and otherwise
    $H(f)=\infty$.
\end{definition}

\begin{lemma}
\label{straightfwdtechnicallemma}
    For every natural $m>0$ and every $f:\mathbb N\to\mathbb N$,
    $H(f)\leq m$ if and only if $f_m\succ f$.
\end{lemma}

\begin{proof}
    Straightforward.
\end{proof}

\begin{definition}
\label{variationondefinitionofEdefn}
    For every $m\in\mathbb N$, let $E^H_m$
    be the set of all evaders $e$ such that $H(t_e)\leq m$.
\end{definition}

\begin{lemma}
\label{equivalenceoftwoevadersetslemma}
    For every natural $m>0$, $E^H_m=E_{f_m}$.
\end{lemma}

\begin{proof}
    Let $e$ be an evader. By Definition \ref{variationondefinitionofEdefn},
    $e\in E^H_m$ if and only if $H(t_e)\leq m$.
    By Lemma \ref{straightfwdtechnicallemma}, $H(t_e)\leq m$ if and only if
    $f_m\succ t_e$. But by Definition \ref{evadersetdefinition}, this is the
    case if and only if $e\in E_{f_m}$.
\end{proof}

\begin{corollary}
\label{rephrasinghibbardsmeasurecorollary}
    For every predictor $p$, the original Hibbard measure of $p$
    is equal to the maximum natural $m>0$ such that
    $p$ learns $e$ whenever $e\in E^H_m$, or is equal to $\infty$
    if $p$ learns $e$ whenever $e\in E^H_m$ for all $m>0$.
\end{corollary}

\begin{proof}
    Immediate by Lemma \ref{equivalenceoftwoevadersetslemma}
    and Definition \ref{classichibbardmeasuredefn}.
\end{proof}

\begin{remark}
\label{epiphanyremark}
Corollary \ref{rephrasinghibbardsmeasurecorollary} shows that
the definition of the original Hibbard measure can be rephrased in such a way
as to show that it depends in a uniform way on a particular solution to
Problem \ref{bigoproblem}, namely to the solution proposed by
Definition \ref{hibbardgrowthratedefn}. For \emph{any} solution $H'$ to
Problem \ref{bigoproblem}, we could define evader-sets $E^{H'}_m$ in a similar
way to Definition \ref{variationondefinitionofEdefn}, and, by copying
Corollary \ref{rephrasinghibbardsmeasurecorollary}, we could obtain a corresponding
intelligence measure given by $H'$ (it might be necessary to replace the
``maximum'' in Corollary \ref{rephrasinghibbardsmeasurecorollary}
by a ``supremum'', if $H'$ measures growth rates using
a non-discrete number system, or transform the form of it if $H'$
solves Problem \ref{bigoproblem} using a non-complete number system or
by categorizing functions into classes
rather than by assigning
them numerical measurements, as in the case of Big-O notation). This
formalizes what we claimed in the Introduction,
that Hibbard's idea can be decomposed into two sub-ideas, firstly, that a predictor's
intelligence can be measured in terms of the growth rates of the runtimes of the
evaders it learns, and secondly, a particular method (Definition \ref{hibbardgrowthratedefn})
of measuring those growth rates (i.e., a particular solution to
Problem \ref{bigoproblem}).
\end{remark}


\section{Big-O, Big-$\Omega$, and Big-$\Theta$ intelligence measurement}
\label{bigosection}

The standard solution to Problem \ref{bigoproblem} in computer science is to quantify
growth rates of arbitrary functions by comparing them to more familiar functions using
Big-O notation, Big-$\Omega$ notation, or Big-$\Theta$ notation.
Knuth defines \cite{knuth1976big} these as follows (we modify the definition slightly because
we are only concerned here with functions from $\mathbb N$ to $\mathbb N$).

\begin{definition}
\label{bigodefn}
    Suppose $f:\mathbb N\to\mathbb N$. We define the following function-sets.
    \begin{itemize}
        \item
        $O(f(n))$ is the set of all $g:\mathbb N\to\mathbb N$ such that
        there is some real $C>0$ and some $n_0\in\mathbb N$ such that
        for all $n\geq n_0$, $g(n)\leq Cf(n)$.
        \item
        $\Omega(f(n))$ is the set of all $g:\mathbb N\to\mathbb N$ such that
        there is some real $C>0$ and some $n_0\in\mathbb N$ such that
        for all $n\geq n_0$, $g(n)\geq Cf(n)$.
        \item
        $\Theta(f(n))$ is the set of all $g:\mathbb N\to\mathbb N$ such that
        there are some real $C>0$ and $C'>0$ and some $n_0\in\mathbb N$ such that
        for all $n\geq n_0$, $Cf(n)\leq g(n)\leq C'f(n)$.
    \end{itemize}
\end{definition}

We would argue that Definition \ref{bigodefn} is a strictly better solution to
Problem \ref{bigoproblem} than Definition \ref{hibbardgrowthratedefn}.
Definition \ref{hibbardgrowthratedefn} would clearly not be much use in
practice. It is much more useful to say that an algorithm has, say, asymptotic
runtime $O(n^3)$ than to say that said algorithm's runtime has original Hibbard
complexity (say) $835$. If someone were told that an algorithm's runtime had
original Hibbard complexity (say) $835$, that number would be totally meaningless
to them except as a key which they could use to look up which function $f_{835}$
happens to be (in Definition \ref{classichibbardmeasuredefn}). The number
$835$ would merely serve to obfuscate, it would merely play the role of a worthless
middleman.

By Remark \ref{epiphanyremark}, Definition \ref{bigodefn} yields the following elegant
measure of predictor intelligence.

\begin{definition}
\label{bigointelligencedefn}
    Suppose $p$ is a predictor, and suppose $f:\mathbb N\to\mathbb N$.
    \begin{itemize}
        \item
        We say \emph{$p$ has Hibbard intelligence $O(f(n))$} if
        $p$ learns every evader $e$ such that $t_e$ is $O(f(n))$.
        \item
        We say \emph{$p$ has Hibbard intelligence $\Omega(f(n))$} if
        $p$ learns every evader $e$ such that $t_e$ is $\Omega(f(n))$.
        \item
        We say \emph{$p$ has Hibbard intelligence $\Theta(f(n))$} if
        $p$ learns every evader $e$ such that $t_e$ is $\Theta(f(n))$.
    \end{itemize}
\end{definition}

At first glance, Definition \ref{bigointelligencedefn} might seem inferior to
Definition \ref{classichibbardmeasuredefn} because
Definition \ref{classichibbardmeasuredefn} assigns \emph{numerical} intelligence levels.
However, as we pointed out above, those numbers are almost meaningless except as indices for
a dictionary-lookup making them equivalent to (a more limited version of)
Definition \ref{bigointelligencedefn}. We could imagine someone declaring:
\begin{quote}
    From now on, $f(n)=n$ is complexity $5$, $f(n)=n^2$ is complexity $8$,
    $f(n)=2^n$ is complexity $203$,
    $f(n)=n!$ is complexity $8022$, ..., and from now on, instead of saying a
    function is $\Theta(n)$, we will say that function has growth rate $5$,
    and instead of saying a function is $\Theta(n^2)$, we will say that function
    has growth rate $8$, and instead of saying a function is $\Theta(2^n)$, we will
    say that function has growth rate $203$, and instead of saying a function is
    $\Theta(n!)$, we shall say that function has growth rate $8022$, and ...
\end{quote}
Superficially, this would ``improve'' upon Big-$\Theta$ by giving a ``more quantitative''
solution to Problem \ref{bigoproblem}, however, it is clear
that this would just be sleight of hand and there would be no actual improvement.

\section{Hibbard measures using extended number systems}
\label{exoticsection}

We have argued above that it is not useful to try to replace asymptotic complexity classes
with natural numbers. In \cite{alexander2020archimedean}, we go further, and cite
asymptotic runtime complexity as an example of a \emph{non-Archimedean} measure which,
we argue, implies that it is not useful to replace asymptotic complexity classes
even with \emph{real} numbers. There are, however, non-Archimedean number systems
flexible enough to measure some or all growth rates, and these alternative solutions to
Problem \ref{bigoproblem} would in turn yield Hibbard-style intelligence measures taking
values from said non-Archimedean number systems.
\begin{itemize}
    \item
    Hyperreal numbers, studied in the field of non-standard analysis \cite{robinson}
    \cite{goldblatt2012lectures},
    are equivalence classes of infinite sequences of reals,
    so for every sequence $(r_0,r_1,r_2,\ldots)$ of reals, there is a corresponding
    hyperreal represented by that sequence.
    Thus, there is a natural and elegant way to solve Problem \ref{bigoproblem}
    using hyperreal numbers. Namely: the growth rate of $f(n)$ is the hyperreal number
    represented by $(f(0),f(1),f(2),\ldots)$. This solution to Problem \ref{bigoproblem}
    will yield an approximate Hibbard-style intelligence measure similar to
    Definition \ref{bigointelligencedefn} (the reason this only yields an approximate
    Hibbard-style intelligence measure is because the hyperreals are not complete).
    \item
    The surreal numbers \cite{conway} \cite{knuth} are an even larger extension
    of the real numbers, into which the hyperreals can be embedded
    \cite{ehrlich2012absolute}.
    Thus, for any such embedding,
    there is another natural and elegant way to solve Problem \ref{bigoproblem},
    namely: the growth rate of $f(n)$ is the surreal number corresponding to the
    hyperreal number represented by $(f(0),f(1),f(2),\ldots)$ under the embedding.
    The advantage of the surreal numbers is that they are complete, which will allow
    the corresponding Hibbard-style measure to be exact, not approximate.
    \item
    Another natural way to measure growth rate is using majorization hierarchies
    (such as the slow-growing hierarchy or the fast-growing hierarchy
    \cite{weiermann2002slow}) from mathematical
    logic. A majorization hierarchy assigns ordinal number values to growth rates of
    functions (but not to all functions---for any particular majorization hierarchy, there
    are functions which grow too fast for that hierarchy). These yield corresponding
    Hibbard-style intelligence measures which are ordinal-number-valued.
    \item
    In order to reduce the arbitrary limitations of traditional majorization
    hierarchies, we could delegate the problem of defining a majorization hierarchy to
    an Artificial General Intelligence (or AGI). This leads to a family of Hibbard-style
    measures---one measure
    $|\bullet|_X$ for each AGI $X$, so that for any particular AGI $X$, for any
    predictor $p$, there is a corresponding intelligence measure $\|p\|_X$ which might be
    thought of as ``$p$'s intelligence as judged by $X$''.
\end{itemize}

\subsection{Hyperreal Hibbard intelligence}

The so-called \emph{ultrapower construction} of the hyperreals depends on an object
called a \emph{free ultrafilter}, which is a set of subsets of $\mathbb N$ satisfying
certain requirements. It is not important for the purposes of this paper to define what
a free ultrafilter is here. The important things to know are:
\begin{itemize}
    \item
    A free ultrafilter $\mathscr U$ provides a notion of what it means for a subset
    $S\subseteq\mathbb N$ to be ``large''. Namely: $S$ is considered to be ``large''
    if and only if $S\in \mathscr U$.
    \item
    Mathematical logicians have proven that free ultrafilters exist, but that it is
    impossible to constructively exhibit one. This makes definitions based on free
    ultrafilters non-constructive, and computationally impractical.
\end{itemize}

Because of the computational impracticality of free ultrafilters, the following notion
is also computationally impractical. However, it could potentially be useful for
proving theoretical properties about the intelligence of predictors. In the following
definition, rather than assigning a particular hyperreal number intelligence to every
predictor, rather, we categorize predictors into classes. This is necessary because
the hyperreals are not complete, so Corollary \ref{rephrasinghibbardsmeasurecorollary}
cannot directly be mimicked either with ``maximum'' or with ``supremum''.

\begin{definition}
\label{hyperrealhibbardintelligencedefn}
    (Hyperreal Hibbard intelligence)
    The \emph{hyperreal Hibbard intelligence} of a predictor $p$ is defined to be
    $\geq$ a hyperreal number $r$ if and only if $p$ learns every evader $e$
    such that if $r'$ is the hyperreal number represented by
    $(t_e(0), t_e(1), t_e(2), \ldots)$ then $r'<r$.
\end{definition}

\subsection{Surreal Hibbard intelligence}

The surreal number system is even
more flexible than the hyperreal number system.
There are many ways to embed the hyperreals into the surreals, and there is no
standard or canonical way which stands out. In this subsection, we fix some embedding
of the hyperreals into the surreals.

A key property of the surreals
is that they are complete. Thus, given any set $S$ of surreals with a surreal upper bound,
there is a \emph{least} surreal upper bound of $S$, called the \emph{supremum} of $S$.
This allows for a Hibbard-style intelligence measure which is more
exact than Definition \ref{hyperrealhibbardintelligencedefn}.

\begin{definition}
\label{surrealhibbardintelligencedefn}
    (Surreal Hibbard intelligence)
    For every predictor $p$, the \emph{surreal Hibbard intelligence} of $p$
    is equal to the supremum of all surreal numbers $r$ such that
    $p$ learns $e$ whenever the surreal number corresponding to
    $(t_e(0),t_e(1),t_e(2),\ldots)$ is $<r$.
\end{definition}

\subsection{Hibbard intelligence and the traditional majorization hierarchies}

Majorization hierarchies \cite{weiermann2002slow}
provide ordinal-number-valued measures for the growth
rate of certain functions. A majorization hierarchy depends
on many infinite-dimensional parameters. For illustrative purposes,
we will describe the slow-growing hierarchy up to the ordinal $\epsilon_0$,
using standard choices for the parameters.

\begin{definition}
    (Classification of ordinal numbers)
    Ordinal numbers are divided into three types:
    \begin{enumerate}
        \item Zero: The ordinal $0$.
        \item Successor ordinals: Ordinals of the form $\alpha+1$ for some ordinal $\alpha$.
        \item Limit ordinals: Ordinals which are not successor ordinals nor $0$.
    \end{enumerate}
\end{definition}

For example, the smallest infinite ordinal, $\omega$, is a limit ordinal. It is not zero
(because zero is finite),
nor can it be a successor ordinal, because if it were a successor ordinal, say, $\alpha+1$,
then $\alpha$ would be finite (since $\omega$ is the \emph{smallest} infinite ordinal),
but then $\alpha+1$ would be finite as well.

The ordinal $\epsilon_0$ is the smallest ordinal bigger than the ordinals
$\omega,\omega^\omega,\omega^{\omega^\omega},\ldots$. It satisfies the equation
$\epsilon_0=\omega^{\epsilon_0}$ and can be intuitively thought of as
\[
    \epsilon_0 = \omega^{\omega^{\omega^{\iddots}}}.
\]
Ordinals below $\epsilon_0$ include such ordinals as $\omega$,
$\omega^{\omega+1}+\omega^{\omega}+\omega^5+3$,
\[
\omega^{\omega^{\omega^{\omega^\omega}}}+
\omega^{\omega^{\omega^\omega}+\omega^{\omega\cdot 2+1}+\omega^4 + 3}
+ \omega^{\omega^5+\omega^3}+\omega^8+1,
\]
and so on.
Any ordinal below $\epsilon_0$ can be uniquely written in the form
\[
    \omega^{\alpha_1}+\omega^{\alpha_2}+\cdots + \omega^{\alpha_k}
\]
where $\alpha_1\geq\cdots\geq\alpha_k$ are smaller ordinals below $\epsilon_0$---this form
for an ordinal below $\epsilon_0$ is called its \emph{Cantor normal form}.
For example, the Cantor normal form for $\omega^{\omega\cdot 2}\cdot 2+\omega\cdot 3+2$
is
\[
\omega^{\omega\cdot 2}\cdot 2+\omega\cdot 3+2
=
\omega^{\omega\cdot 2} + \omega^{\omega\cdot 2} + \omega^1 + \omega^1 + \omega^1
+\omega^0 + \omega^0.
\]

\begin{definition}
    (Standard fundamental sequences for limit ordinals $\leq\epsilon_0$)
    Suppose $\lambda$ is a limit ordinal $\leq\epsilon_0$. We define a
    \emph{fundamental sequence for $\lambda$},
    written $(\lambda[0],\lambda[1],\lambda[2],\ldots)$, inductively as follows.
    \begin{itemize}
        \item
        If $\lambda=\epsilon_0$, then $\lambda[0]=0$,
        $\lambda[1]=\omega^0$, $\lambda[2]=\omega^{\omega^0}$, and so on.
        \item
        If $\lambda$ has Cantor normal form
        $\omega^{\alpha_1}+\cdots+\omega^{\alpha_k}$ where $k>1$,
        then
        each
        \[
            \lambda[i] = (\omega^{\alpha_1}+\cdots+\omega^{\alpha_{k-1}})
            + (\omega^{\alpha_k})[i].
        \]
        \item
        If $\lambda$ has Cantor normal form $\omega^{\alpha+1}$,
        then each $\lambda[i]=\omega^{\alpha}\cdot i$.
        \item
        If $\lambda$ has Cantor normal form $\omega^{\lambda_0}$ where $\lambda_0$
        is a limit ordinal, then each $\lambda[i]=\omega^{\lambda_0[i]}$.
    \end{itemize}
\end{definition}

\begin{example}
    (Fundamental sequence examples)
    \begin{itemize}
        \item
        The fundamental sequence for $\lambda=\omega=\omega^1=\omega^{0+1}$ is
        $\omega^0\cdot 0, \omega^0\cdot 1, \omega^0\cdot 2, \ldots$,
        i.e., $0, 1, 2, \ldots$.
        \item
        The fundamental sequence for $\lambda=\omega^5$ is
        $0,\omega^4,\omega^4\cdot 2,\omega^4\cdot 3,\ldots$.
        \item
        The fundamental sequence for $\lambda=\omega^\omega$ is
        $\omega^0,\omega^1,\omega^2,\ldots$.
        \item
        The fundamental sequence for $\lambda=\omega^\omega+\omega$ is
        $\omega^\omega+0,\omega^\omega+1,\omega^\omega+2,\ldots$.
    \end{itemize}
\end{example}

\begin{definition}
\label{slowgrowinghierarchydefn}
    (The standard slow-growing hierarchy up to $\epsilon_0$)
    We define functions $g_\beta:\mathbb N\to\mathbb N$ (for all ordinals
    $\beta\leq \epsilon_0$) by transfinite induction as follows.
    \begin{itemize}
        \item
        $g_0(n)=0$.
        \item
        $g_{\alpha+1}(n) = g_\alpha(n) + 1$.
        \item
        $g_{\lambda}(n) = g_{\lambda[n]}(n)$ if $\lambda$ is a limit ordinal.
    \end{itemize}
\end{definition}

Here are some early levels in the slow-growing hierarchy, spelled out in detail.

\begin{example}
\label{highdetailslowgrowingexample}
    (Early examples of functions in the slow-growing hierarchy)
    \begin{enumerate}
        \item
        $g_1(n)=g_{0+1}(n)=g_0(n)+1=0+1=1$.
        \item
        $g_2(n)=g_{1+1}(n)=g_1(n)+1=1+1=2$.
        \item
        More generally, for all $m\in\mathbb N$,
        $g_m(n)=m$.
        \item
        $g_\omega(n)=g_{\omega[n]}(n)=g_n(n)=n$.
        \item
        $g_{\omega+1}(n)=g_{\omega}(n)+1=n+1$.
        \item
        $g_{\omega+2}(n)=g_{\omega+1}(n)+1=(n+1)+1=n+2$.
        \item
        More generally, for all $m\in\mathbb N$,
        $g_{\omega+m}(n)=n+m$.
        \item
        $g_{\omega\cdot 2}(n)=g_{(\omega\cdot 2)[n]}(n)
        =g_{\omega+n}(n)=n+n=n\cdot 2$.
    \end{enumerate}
\end{example}

Following Example \ref{highdetailslowgrowingexample}, the reader should be able
to fill in the details in the following example.

\begin{example}
    (More examples from the slow-growing hierarchy)
    \begin{enumerate}
        \item
        $g_{\omega^2}(n)=n^2$.
        \item
        $g_{\omega^3}(n)=n^3$.
        \item
        $g_{\omega^\omega}(n)=n^n$.
        \item
        $g_{\omega^{\omega\cdot 3+1}+\omega+5}(n)=n^{3n+1}+n+5$.
        \item
        $g_{\omega^{\omega^{\omega}}}(n)=n^{n^n}$.
    \end{enumerate}
\end{example}

What about $g_{\epsilon_0}$? Thinking of $\epsilon_0$ as
\[\omega^{\omega^{\omega^{\iddots}}},\]
one might expect $g_{\epsilon_0}(n)$ to be
\[n^{n^{n^{\iddots}}},\]
but such an infinite tower
of exponents makes no sense if $n>1$. Instead, the answer defies familiar mathematical
notation.

\begin{example}
\label{epsilon0example}
(Level $\epsilon_0$ in the slow-growing hierarchy)
The values of $g_{\epsilon_0}$ are as follows:
\begin{itemize}
    \item
    $g_{\epsilon_0}(0)=0$.
    \item
    $g_{\epsilon_0}(1)=1^1$.
    \item
    $g_{\epsilon_0}(2)=2^{2^2}$.
    \item
    $g_{\epsilon_0}(3)=3^{3^{3^3}}$.
    \item
    And so on.
\end{itemize}
\end{example}

Example \ref{epsilon0example} illustrates how the slow-growing hierarchy can systematically
lead to fast-growing computable functions. By extending the slow-growing hierarchy to
larger ordinals (and choosing appropriate fundamental sequences for those larger ordinals),
one can obtain stupendously fast-growing functions in this way. These in turn serve as
a partial solution to Problem \ref{bigoproblem}: we can say that the growth rate of
an arbitrary function $f:\mathbb N\to\mathbb N$
is equal to $\alpha$, where $\alpha$ is the smallest ordinal such
that $g_\alpha\succ f$ (or $\infty$ if no such $\alpha$ exists).
This in turn provides a corresponding Hibbard-style measure.

\begin{definition}
\label{tradmajorizationhierarchyhibbardmeasuredefn}
    If $p$ is a predictor, the \emph{Hibbard intelligence of $X$ given by the
    standard slow-growing hierarchy up to $\epsilon_0$} is defined to be the
    supremum of all ordinals $\alpha<\epsilon_0$ such that
    $p$ learns every evader $e$ such that $g_\alpha\succ t_e$ (or to
    $\infty$ if said property holds for all ordinals $\alpha<\epsilon_0$).
\end{definition}

\subsection{Generalized majorization hierarchies}

There are many other majorization hierarchies besides the slow-growing hierarchy.
In Definition \ref{slowgrowinghierarchydefn},
there is nothing special about defining $g_{\alpha+1}(n)=g_{\alpha}(n)+1$.
Many other definitions for $g_{\alpha+1}$ would work, provided $g_{\alpha+1}$ ends
up being faster-growing than $g_\alpha$.
For example, in the so-called \emph{fast-growing hierarchy},
$g_{\alpha+1}(n)$ is defined to be $g^n_\alpha(n)$, where $g^n_\alpha$
denotes the result of iterating $g_\alpha$ $n$ times, that is,
$g^1_\alpha(n)=g_\alpha(n)$, $g^2_\alpha(n)=g_\alpha(g_\alpha(n))$,
$g^3_\alpha(n)=g_\alpha(g_\alpha(g_\alpha(n)))$, and so on.
In the early levels, this produces much faster-growing functions,
but astonishingly, at sufficiently high ordinals, the slow-growing hierarchy
actually catches up with the fast-growing hierarchy \cite{girard1981pi12}
(a beautiful illustration of how counter-intuitive large ordinal numbers can be).
One of the earliest majorization hierarchies is the Hardy
hierarchy \cite{hardy1904theorem}, where $g_{\alpha+1}(n)=g_\alpha(n+1)$.

Likewise, there is also much flexibility in choosing fundamental sequences.
For small ordinals such as the ordinals below $\epsilon_0$, there are very clear
canonical fundamental sequences, but the larger the ordinals become, the harder
it becomes to single out any choice of fundamental sequences as canonical.
And even at low levels, choosing non-canonical fundamental sequences can
drastically alter the resulting majorization hierarchy \cite{weiermann1997sometimes}.

In short, there is no consensus at all about which majorization hierarchies are
most canonical. If anything, there is consensus that there is no consensus.
However, Hibbard-style intelligence measures are primarily of interest to researchers
interested in agents with Artificial General Intelligence (that is, researchers
interested in AGIs). No-one knows what exactly an AGI is, but presumably an AGI can
be thought of\footnote{Our thinking here is reminiscent of some remarks of
Yampolskiy \cite{yampolskiy2013turing}.} as a patient, obedient, careful,
mechanical employee who can
be given commands in English, and who will follow those commands (at least when
the commands are \emph{possible} to follow---one could command an AGI to act as a
Halting Problem solver, but it is unclear how the AGI would respond to such a command,
since no mechanical agent can solve the Halting Problem).

Rather than attempt the futile task of choosing a canonical majorization hierarchy
ourselves, we can instead delegate that task to an AGI. By doing so for an arbitrary
AGI, we will obtain an AGI-indexed family of semi-canonical majorization hierarchies.

\begin{definition}
\label{fairsequencedefn}
    Suppose $X$ is an AGI. By $h^X_1,h^X_2,\ldots$, we mean the total computable
    increasing functions
    from $\mathbb N$ to $\mathbb N$ which $X$ would output if $X$ were commanded:
    \begin{quote}
        ``Until further notice, output (codes of) total computable increasing
        functions $h_1,h_2,\ldots$ from $\mathbb N$ to $\mathbb N$ satisfying the
        following constraints:
        \begin{enumerate}
            \item (Linear ordering) For any two $i,j\in\mathbb N$ with $i\neq j$, either
            $h_i\succ h_j$ or $h_j\succ h_i$.
            \item (Largeness) For every Turing machine $M$ such that you know
            $M$ computes a total computable function $h:\mathbb N\to\mathbb N$,
            there is some $i$ such that $h_i\succ h$.
            \item (Well-foundedness) There is no sequence $i_1,i_2,\ldots$ such that
            $h_{i_1}\succ h_{i_2}\succ\cdots$.
            \item (Pseudo-density)
            The $h_i$'s are to be as close as possible to being \emph{dense}---that is
            to say, they are to be as close as possible to having the property that
            for all $i,k\in\mathbb N$ with $h_k\succ h_i$, there is some $j\in\mathbb N$
            such that $h_k\succ h_j\succ h_i$---without violating the above
            well-foundedness constraint; you are to use your judgment and discretion
            to interpret this pseudo-density requirement.''
        \end{enumerate}
    \end{quote}
\end{definition}

The above pseudo-density constraint is intentionally vague, which we can get away
with because the definition does not depend on what pseudo-density actually means,
but only on how the AGI interprets the \emph{words} (i.e., how the AGI responds to a
specific well-defined stimulus, regardless of how vague the semantics of that stimulus
may be).
Thus, we are taking advantage of the AGI's ability to ``adapt with insufficient
knowledge and resources'' \cite{wang2019defining}.
We assume the AGI understands and obeys the command, so $h^X_1,h^X_2,\ldots$ really are
total computable increasing functions from $\mathbb N$ to $\mathbb N$ satisfying the
linear ordering, largeness\footnote{Note that the largeness requirement is defined in
terms of the AGI's mathematical
knowledge. See \cite{alexanderinprep} for a definition of what
it means for an AGI to know a mathematical fact.}, and well-foundedness requirements.
Note that the resulting $h_i$'s
cannot actually be dense, as that would violate well-foundedness: given any $i,j$
with $h_j\succ h_i$, there would be some $k_1$ such that $h_j\succ h_{k_1}\succ h_i$,
and then there would be some $k_2$ such that $h_{k_1}\succ h_{k_2}\succ h_i$,
and then there would be some $k_3$ such that $h_{k_2}\succ h_{k_3}\succ h_i$, and so
on, but then $h_{k_1}\succ h_{k_2}\succ \cdots$ would violate well-foundedness.
We will offer some motivation for pseudo-density in
Remark \ref{pseudodensityremark} below.

\begin{definition}
\label{alphaidefn}
    Let $X$ be an AGI. We define a sequence $\alpha^X_1,\alpha^X_2,\ldots$
    of ordinals by recursion as follows. Note that at first glance,
    the following recursive definition looks too circular to work.
    We will show in Lemma \ref{transfiniterecursionmagiclemma} that
    it avoids infinite regress and thus
    it does work. For each $i$, let $\alpha^X_i$ be the the smallest
    ordinal which is larger than every ordinal $\alpha^X_j$ such that
    $h^X_i\succ h^X_j$.
\end{definition}

\begin{lemma}
\label{transfiniterecursionmagiclemma}
    Definition \ref{alphaidefn} works (it does not lead to infinite regress).
\end{lemma}

\begin{proof}
    Assume, for the sake of contradiction, that Definition \ref{alphaidefn}
    leads to infinite regress. This means there is some $i_1$ such that
    in order to define $\alpha^X_{i_1}$ we must first define $\alpha^X_{i_2}$
    for some $i_2$, and in order to define $\alpha^X_{i_2}$ we must first
    define $\alpha^X_{i_3}$ for some $i_3$, and in order to define
    $\alpha^X_{i_3}$ we must first define $\alpha^X_{i_4}$ for some $i_4$,
    and so on forever. Thus, there is an infinite sequence $i_1,i_2,\ldots$
    such that in order to define each $\alpha^X_{i_j}$, we must first
    define $\alpha^X_{i_{j+1}}$. Now, each $\alpha^X_{i_j}$ is defined as
    the smallest ordinal larger than every ordinal $\alpha^X_j$ such that
    $h^X_{i_j}\succ h^X_j$. So, since defining $\alpha^X_{i_j}$ requires
    us to first define $\alpha^X_{i_{j+1}}$, this implies $i_{j+1}$ is
    such a $j$, i.e., that $h^X_{i_j}\succ h^X_{i_{j+1}}$.
    Thus $h^X_{i_1}\succ h^X_{i_2}\succ h^X_{i_3}\succ\cdots$,
    but this contradicts the \emph{well-foundedness} part of the command
    from Definition \ref{fairsequencedefn}.
\end{proof}

\begin{lemma}
\label{technicallemmaaboutalphaXi}
    Let $X$ be an AGI. For all $i,j$,
    $h^X_i\succ h^X_j$ if and only if $\alpha^X_i>\alpha^X_j$.
\end{lemma}

\begin{proof}
    Fix $i$ and $j$.
    By the \emph{linear ordering} part of the command in Definition
    \ref{fairsequencedefn}, either $h^X_i\succ h^X_j$ or $h^X_j\succ h^X_i$.
    Assume $h^X_i\succ h^X_j$, the other case is similar.
    By definition, $\alpha^X_i$ is defined to be the smallest ordinal
    which is larger than $\alpha^X_{j'}$ for every $j'$ such that
    $h^X_i\succ h^X_{j'}$. One such $j'$ is $j$ itself,
    so by definition $\alpha^X_i$ is larger than $\alpha^X_j$, as desired.
\end{proof}

\begin{definition}
\label{generalizedmajorizationhierarchydefn}
(Generalized majorization hierarchies)
    Let $X$ be an AGI. Let $\alpha=\sup_i \alpha^X_i$ be the smallest ordinal
    bigger than all of the $\alpha^X_i$'s. The \emph{generalized majorization
    hierarchy given by $X$} is the family $(g_\beta)_{\beta<\alpha}$ of functions
    labeled by ordinals below $\alpha$, where each $g_{\beta}$ is defined
    such that $g_{\beta}=h^X_i$ where $i$ is such that $\beta=\alpha^X_i$.
\end{definition}

\begin{remark}
\label{pseudodensityremark}
With Definition \ref{generalizedmajorizationhierarchydefn} in view,
the pseudo-density
constraint in Definition \ref{fairsequencedefn} can be better motivated:
the point of pseudo-density is to ensure that the ordinal
$\sup_i \alpha^X_i$ is as large as possible\footnote{Chaitin \cite{chaitin}
and Good \cite{good1969godel} have remarked
about the importance of the size of the ordinals which an entity can notate---``and
the larger the ordinal,
the fitter the organism'', to quote Chaitin. See also \cite{ioi1} and \cite{ioi2}.}. An
alternative to pseudo-density would
be to directly command (in Definition \ref{fairsequencedefn}) that the $h_i$'s should
be chosen in such a way as to make $\alpha=\sup_i\alpha^X_i$ large (leaving it up
to the AGI's judgment and discretion how to interpret that), but this would be
difficult because it would mean that Definition \ref{generalizedmajorizationhierarchydefn}
(and its dependencies) would need to be embedded into the command in
Definition \ref{fairsequencedefn}, which would require some significant acrobatics.
\end{remark}

A generalized majorization hierarchy (Definition \ref{generalizedmajorizationhierarchydefn})
serves as a solution to Problem \ref{bigoproblem}: given any AGI $X$, we can say
that a function $f:\mathbb N\to\mathbb N$ has growth rate
$\beta$ where $\beta$ is the smallest ordinal $<\sup_i \alpha^X_i$
such that $g_\beta\succ f$---or $f$ has growth rate $\infty$ if there is no
such $\beta$. To this solution to Problem \ref{bigoproblem}, there is a corresponding
Hibbard-style intelligence measure.

\begin{definition}
    (Hibbard intelligence given by an AGI)
    Suppose $X$ is an AGI. By the \emph{transfinite Hibbard measure given by $X$},
    we mean the measure $\|\bullet\|_X$ which assigns to every predictor $p$ an
    intelligence measure $\|p\|_X$ defined as follows.
    $\|p\|_X$ is defined to be the smallest $\beta<\sup_i\alpha^X_i$
    such that for all $\gamma<\beta$,
    $p$ learns every evader $e$ such that $g_\gamma\succ e$
    (or $\|p\|_X=\infty$ if
    $p$ learns every evader $e$ such that $g_\gamma\succ e$ for every
    $\gamma<\sup_i \alpha^X_i$).
\end{definition}

\section{Pros and cons of different Hibbard-style measures}
\label{prosandconssection}

Here are pros and cons of the Hibbard measures which arise from different solutions
to the problem (Problem \ref{bigoproblem}) of measuring the growth rate of functions.

\begin{itemize}
    \item
    The original Hibbard measure (Definition \ref{classichibbardmeasuredefn}),
    which arises by measuring growth rate by comparing
    a function with one particular enumeration \cite{liu1960enumeration} of the primitive
    recursive functions:
    \begin{itemize}
        \item
        Pro: Relatively concrete.
        \item
        Pro: Measures intelligence using a familiar number system (the natural numbers).
        \item
        Con: The numbers which the measure outputs are not very meaningful, in the sense
        that predictor $p$ having a measure of
        just $+1$ higher than predictor $q$ tells us absolutely
        nothing about how \emph{much} more computationally complex the evaders which $p$
        learns are, versus the evaders which $q$ learns.
        \item
        Con: Only distinguishes sufficiently non-intelligence predictors; all predictors
        sufficiently intelligent receive measure $\infty$.
    \end{itemize}
    \item
    Big-O/Big-$\Theta$/Big-$\Omega$ (Definition \ref{bigointelligencedefn}),
    in which, rather than directly measuring the intelligence of a predictor, instead, we
    would talk of a predictor's intelligence being $O(f(n))$, or $\Theta(f(n))$, or
    $\Omega(f(n))$, for various functions $f:\mathbb N\to\mathbb N$:
    \begin{itemize}
        \item
        Pro: Gets directly at the underlying concept, without obfuscation.
        \item
        Pro: Computer scientists already use Big-O/Big-$\Theta$/Big-$\Omega$ routinely
        and are comfortable with them.
        \item
        Con: This option is not really a measure, but more of a taxonomy---and a non-numerical
        taxonomy at that.
    \end{itemize}
    \item
    Hyperreal intelligence (Definition \ref{hyperrealhibbardintelligencedefn}):
    \begin{itemize}
        \item
        Pro: A taxonomy like Big-O/Big-$\Theta$/Big-$\Omega$, but with the added benefit
        that the taxons are numerical (hyperreal numerical, to be more precise).
        \item
        Con: Depends on a non-constructive choice of a free ultrafilter (rendering it
        impractical for any actual computation).
    \end{itemize}
    \item
    Surreal intelligence (Definition \ref{surrealhibbardintelligencedefn}):
    \begin{itemize}
        \item
        Pro: An actual numerical measure (not just a taxonomy), with the same perfect
        precision as the Big-O/Big-$\Theta$/Big-$\Omega$ approach.
        \item
        Con: Abstract and impractical: depends not only on a
        non-constructive choice of
        a free ultrafilter, but also on an embedding of the hyperreals into the surreals.
        \item
        Con: The numbers which the measure outputs are surreal numbers, which may be
        unfamiliar to some users.
    \end{itemize}
    \item
    Intelligence based on a standard majorization hierarchy such as the
    standard slow-growing hierarchy up to $\epsilon_0$
    (Definition \ref{tradmajorizationhierarchyhibbardmeasuredefn}):
    \begin{itemize}
        \item
        Pro: A numerical measure, albeit without as much precision as the
        Big-O/Big-$\Theta$/Big-$\Omega$ taxonomy.
        \item
        Pro: Relatively concrete.
        \item
        Pro: The numbers which the measure outputs are meaningful, in the sense that
        the degree to which a predictor $p$ is more intelligent than a
        predictor $q$ is reflected
        in the degree to which $p$'s intelligence-measure is larger than $q$'s.
        \item
        Con: The numbers which the measure outputs are ordinal numbers, which may be
        unfamiliar to some users.
        \item
        Con: Only distinguishes sufficiently non-intelligence predictors; all predictors
        sufficiently intelligent receive measure $\infty$.
    \end{itemize}
    \item
    Intelligence based on a generalized majorization hierarchy
    (Definition \ref{generalizedmajorizationhierarchydefn}):
    \begin{itemize}
        \item
        Pro: A numerical measure, albeit without as much precision as the
        Big-O/Big-$\Theta$/Big-$\Omega$ taxonomy.
        \item
        Pro: The numbers which the measure outputs are meaningful, in the sense that
        the degree to which a predictor $p$ is more
        intelligent than a predictor $q$ is reflected
        in the degree to which $p$'s intelligence-measure is larger than $q$'s.
        \item
        Pro: Potentially capable of assigning non-$\infty$ intelligence measures to
        very advanced predictors (the more intelligent the AGI used to generate the
        generalized majorization hierarchy, the more advanced the predictors can be
        and still receive non-$\infty$ intelligence).
        \item
        Con: The numbers which the measure outputs are ordinal numbers, which may be
        unfamiliar to some users.
        \item
        Con: Actually realizing one of these measures would require access to an AGI.
    \end{itemize}
\end{itemize}

\section{Summary and Conclusion}
\label{conclusionsection}

To summarize:
\begin{itemize}
    \item
    Hibbard proposed \cite{hibbard} an intelligence measure for predictors
    in games of adversarial sequence prediction.
    \item
    We argued that Hibbard's idea actually splits into two orthogonal sub-ideas.
    Firstly: that intelligence can be measured via the growth-rate of the run-times
    of evaders that a predictor can learn to predict. Secondly: that such growth-rate can
    be measured in one specific way (involving an enumeration of the primitive recursive
    functions). We argue that there are other, more standard ways to measure growth-rates,
    and that each method of measuring growth-rates yields a corresponding Hibbard-style
    intelligence measure.
    \item
    We considered several specific ways of measuring growth-rate of functions, and exhibited
    corresponding Hibbard-style intelligence measures. The growth-rate-measuring methods
    which we considered were: Big-O/Big-$\Theta$/Big-$\Omega$ notation; hyperreal numbers;
    surreal numbers; standard majorization hierarchies; and (in their debut appearance here)
    generalized majorization hierarchies.
\end{itemize}


\bibliographystyle{plain}
\bibliography{hibbard}
\end{document}
