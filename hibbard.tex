\documentclass{article}
\usepackage[utf8]{inputenc}
\usepackage{natbib}
\usepackage{amssymb}
\usepackage{amsthm}
\newtheorem{definition}{Definition}

\title{Transfinite extensions of Hibbard's intelligence measure}
\author{Samuel Alexander\thanks{The U.S.\ Securities and Exchange Commission}}
\date{2020}

\begin{document}

\maketitle

\begin{abstract}
    Fill this in.
\end{abstract}

\section{Introduction}

In his insightful paper \cite{hibbard}, Bill Hibbard introduces a novel
intelligence measure (which we will here refer to as the \emph{classical Hibbard measure})
for agents with Artificial General Intelligence (or AGIs).
Hibbard's measure is based on ``the game of adversarial sequence prediction
against a hierarchy of increasingly difficult sets of'' evaders (environments that attempt
to emit $1$s and $0$s in such a way as to evade prediction).
The levels of Hibbard's hierarchy are labelled by natural numbers\footnote{Technically,
Hibbard's hierarchy begins at level $1$ and he separately defines what it means for
an agent to have intelligence $0$, but that definition is equivalent to what would result
by declaring that the $0$th level of the hierarchy consists of the empty set of evaders.}, and
an agent's classical Hibbard measure is the maximum $n\in\mathbb N$ such that
said agent can eventually predict all the evaders in the $n$th level of the hierarchy,
or implicitly\footnote{Hibbard does not explicitly include the $\infty$ case in his
definition, but in his Proposition 3 he refers to agents having ``finite intelligence'', and
it is clear from context that by this he means agents who fail to predict some evader
somewhere in the hierarchy.} an agent's classical Hibbard measure is defined to be $\infty$
if said agent can eventually predict all the evaders in all levels of the hierarchy.

In this paper, we will argue that the evader-hierarchy which Hibbard originally proposed is
too small: we will argue that \emph{all} suitably idealized AGIs have $\infty$
intelligence according to the classical Hibbard measure. In order to remedy this
situation, we will propose a family of large evader-hierarchies. Each
evader-hierarchy in this family will admit a corresponding intelligence measure.
Whereas the classical Hibbard
measure is natural-number-valued, the intelligence measures obained from these larger
evader-hierarchies will be computable-ordinal-number-valued. Larger evader-hierarchies
will admit better intelligence measures, in the sense that the larger the evader-hierarchy,
the more difficult it will be for an AGI to have intelligence $\infty$ (i.e., for an
AGI to eventually predict all the evaders in all levels of said evader-hierarchy).

Our enlarged evader-hierarchies will be constructed using a technique
closely related to the so-called
\emph{majorization hierarchies} from mathematical logic \cite{weiermann2002slow}.
However, the majorization hierarchies in mathematical logic depend on so-called
\emph{fundamental sequences} which (in the long run) have problems both in
theory and practice, and so we modify the technique to be more concrete. Our modified
technique depends on what we call \emph{Intuitive Ordinal Notations}, which are more
concrete and practical (in the long run).

Finally, we will argue that, when applied to AGIs satisfying certain assumptions,
these extended Hibbard intelligence measures
are closely related to the so-called \emph{Intuitive Ordinal Intelligence} measure
which we introduced in \cite{ioi1} and \cite{ioi2}.

The structure of the paper is as follows.
\begin{itemize}
    \item
    In Section \ref{originalmeasuresection}, we review the classical Hibbard measure.
    \item
    In Section ..., we present our viewpoint of AGIs and argue that
    under certain idealizing assumptions, \emph{every} AGI should have intelligence
    $\infty$ according to the classical Hibbard measure.
    \item
    In Section ..., we review Intuitive Ordinal Notations.
    \item
    In Section ..., we describe a technique for constructing fast-growing functions
    $f:\mathbb N\to\mathbb N$ from Intuitive Ordinal Notations. This technique is
    closely related to the \emph{majorization hierarchies} from mathematical logic.
    \item
    In Section ..., we introduce a family of large evader-hierarchies and corresponding
    Hibbard intelligence measures.
    \item
    In Section ..., we discuss relationships between Hibbard intelligence measures
    and Intuitive Ordinal Intelligence.
\end{itemize}

\section{Hibbard's original measure}
\label{originalmeasuresection}

Hibbard's intelligence measure is based on predictors and evaders---the predictor
representing the AGI whose intelligence we would like to measure, and the evader
representing an environment---which we define below. A predictor and an evader
are thought of as interacting together in a ``game of adversarial sequence prediction''.

\begin{definition}
By $B$, we mean the binary alphabet $\{0,1\}$. By $B^*$, we mean the set of all
finite binary sequences. By $B^\infty$, we mean the set of all infinite binary
sequences. By $\langle\rangle$ we mean the empty binary sequence.
\end{definition}

\begin{definition}
    (Predictors and evaders)
    \begin{enumerate}
        \item
        By an \emph{evader}, we mean a function $e:B^*\to B$
        which takes as input a finite (possibly empty) empty sequence $(y_0,\ldots,y_n)$
        (thought of as a sequence of \emph{predictions})
        and outputs $0$ or $1$ (thought of as an \emph{evasion}).
        \item
        By a \emph{predictor}, we mean a function $p:B^*\to B$
        which takes as input a finite (possibly empty) binary sequence $(x_0,\ldots,x_n)$
        (thought of as a sequence of \emph{evasions})
        and outputs $0$ or $1$ (thought of as a \emph{prediction}).
        \item
        For any evader $e$ and predictor $p$, the \emph{result of $p$ playing the
        game of adversarial prediction against $e$} (or more simply, the \emph{result of
        $p$ playing against $e$}) is the infinite binary sequence
        $(x_0,y_0,x_1,y_1,\ldots)\in B^\infty$
        defined as follows:
        \begin{enumerate}
            \item
            $x_0=e(\langle\rangle)$ is obtained by plugging the empty prediction-sequence into
            the evader $e$. This is thought of as $e$'s initial evasion.
            \item
            $y_0=p(\langle\rangle)$ is obtained by plugging the empty evasion-sequence into
            the predictor $p$. This is thought of as $p$'s initial prediction.
            \item
            For all $n\in\mathbb N$,
            $x_{n+1}=e(y_0,\ldots,y_n)$ is obtained by plugging the first $n+1$ predictions
            into the evader $e$. This is thought of as $e$'s $(n+1)$th evasion.
            \item
            For all $n\in\mathbb N$,
            $y_{n+1}=p(x_0,\ldots,x_n)$ is obtained by plugging the first $n+1$ evasions into
            the predictor $p$. This is thought of as $p$'s $(n+1)$th prediction.
        \end{enumerate}
        \item
        Suppose $r=(x_0,y_0,x_1,y_1,\ldots)$ is the result of a predictor $p$ playing
        against an evader $e$. For every $n\in\mathbb N$, the \emph{winner of round $n$}
        in $r$ is $p$ if $x_n=y_n$, or is $e$ if $x_n\neq y_n$.
        We say that \emph{$p$ learns to predict $e$} if there is some $N\in\mathbb N$
        such that for all $n>N$, $p$ is the winner of round $n$.
    \end{enumerate}
\end{definition}

\bibliographystyle{plain}
\bibliography{hibbard}
\end{document}
